\section{User Interface\label{sec:userinterface}}
\markboth{\textsc{MLD2P4 User's and Reference Guide}}
         {\textsc{\ref{sec:userinterface} User Interface}}


The basic user interface of MLD2P4 consists of six routines. The four routines \verb|mld_| \verb|precinit|,
\verb|mld_precset|, \verb|mld_precbld| and \verb|mld_precaply| encapsulate all the functionalities
for the setup and the application of any one-level and multi-level
preconditioner implemented in the package.
The routine \verb|mld_precfree| deallocates the preconditioner data structure, while
\verb|mld_precdescr| prints a description of the preconditioner setup by the user.

For each routine, the same user interface is overloaded with
respect to the real/complex case and the single/double precision;
arguments with appropriate data types must be passed to the routine,
i.e.
\begin{itemize}
\item the sparse matrix data structure, containing the matrix to be
  preconditioned, must be of type \verb|mld_|\emph{x}\verb|spmat_type|
	with \emph{x} = \verb|s| for real single precision, \emph{x} = \verb|d|
	for real double precision, \emph{x} = \verb|c| for complex single precision,
	\emph{x} = \verb|z| for complex double precision;
\item the preconditioner data structure must be of type
  \verb|mld_|\emph{x}\verb|prec_type|, with \emph{x} =    
  \verb|s|, \verb|d|, \verb|c|, \verb|z|, according to the sparse
  matrix data structure;
\item the arrays containing the vectors $v$ and $w$ involved in
  the preconditioner application $w=M^{-1}v$ must be of type   
  \emph{type}\verb|(|\emph{kind\_parameter}\verb|)|, with \emph{type} =
  \verb|real|, \verb|complex| and \emph{kind\_parameter} = \verb|kind(1.e0)|,
  \verb|kind(1.d0)|, according to the sparse matrix and preconditioner
  data structure; note that the PSBLAS module \verb|psb_base_mod|
  provides the constants \verb|psb_spk_|
  = \verb|kind(1.e0)| and \verb|psb_dpk_| = \verb|kind(1.d0)|;
\item real parameters defining the preconditioner must be declared
  according to the precision of the sparse matrix and preconditioner
  data structures (see Section~\ref{sec:precset}).
\end{itemize}
A description of each routine is given in the remainder of this section.

\clearpage

\subsection{Subroutine mld\_precinit\label{sec:precinit}}

\begin{center}
\verb|mld_precinit(p,ptype,info)| \\
\verb|mld_precinit(p,ptype,info,nlev)| \\
\end{center}

\noindent
This routine allocates and initializes the preconditioner data structure,
according to the preconditioner type chosen by the user.

{\vskip2\baselineskip\noindent\large\bfseries Arguments}

\begin{tabular}{p{1.2cm}p{12cm}}
\verb|p|      & \verb|type(mld_|\emph{x}\verb|prec_type), intent(inout)|.\\
              & The preconditioner data structure. Note that \emph{x}
                must be chosen according to the real/complex, single/double
                precision version of MLD2P4 under use.\\
\verb|ptype|  & \verb|character(len=*), intent(in)|.\\
              & The type of preconditioner. Its values are specified
              in Table~\ref{tab:precinit}.\\
              & Note that the strings are case insensitive.\\
\verb|info|   & \verb|integer, intent(out)|.\\
              & Error code. If no error, 0 is returned. See Section~\ref{sec:errors} for details.\\
\verb|nlev|   & \verb|integer, optional, intent(in)|.\\
              & The number of levels of the multilevel preconditioner.
                If \verb|nlev| is not present and \verb|ptype|=\verb|'ML'|, \verb|'ml'|, 
                then \verb|nlev|=2 is assumed. Otherwise, \verb|nlev| is ignored.\\
\end{tabular}

\clearpage

\subsection{Subroutine mld\_precset\label{sec:precset}}

\begin{center}
\verb|mld_precset(p,what,val,info)|\\
\end{center}

\noindent
This routine sets the parameters defining the preconditioner. More
precisely, the parameter identified by \verb|what| is assigned the value
contained in \verb|val|.

{\vskip2\baselineskip\noindent\large\bfseries Arguments}

\begin{tabular}{p{1.2cm}p{12cm}}
\verb|p|      & \verb|type(mld_|\emph{x}\verb|prec_type), intent(inout)|.\\
              & The preconditioner data structure. Note that \emph{x} must
                be chosen according to the real/complex, single/double precision
                 version of MLD2P4 under use.\\
\verb|what|   & \verb|integer, intent(in)|. \\
              & The number identifying the parameter to be set.
                A mnemonic constant has been associated to each of these
                numbers, as reported in Tables~\ref{tab:p_type}-\ref{tab:p_coarse}.\\
\verb|val |   & \verb|integer| \emph{or} \verb|character(len=*)| \emph{or}
                \verb|real(psb_spk_)| \emph{or} \verb|real(psb_dpk_)|,
                \verb|intent(in)|.\\
              & The value of the parameter to be set. The list of allowed
                values and the corresponding data types is given in
                Tables~\ref{tab:p_type}-\ref{tab:p_coarse}.
                When the value is of type \verb|character(len=*)|,
                it is also treated as case insensitive.\\
\verb|info|   & \verb|integer, intent(out)|.\\
              & Error code. If no error, 0 is returned. See Section~\ref{sec:errors}
                for details.\\
%
%\verb|ilev|   & \verb|integer, optional, intent(in)|.\\
%              & For the multilevel preconditioner, the level at which the
%                preconditioner parameter has to be set.
%                The levels are numbered in increasing
%                order starting from the finest one, i.e.\ level 1 is the finest level.
%                If \verb|ilev| is not present, the parameter identified by \verb|what|
%                is set at all the appropriate levels (see Table~\ref{tab:params}).
\end{tabular}

\ \\
A variety of (one-level and multi-level) preconditioners can be obtained
by a suitable setting of the preconditioner parameters. These parameters
can be logically divided into four groups, i.e.\ parameters defining
\begin{enumerate}
	\item the type of multi-level preconditioner;
	\item the one-level preconditioner used as smoother;
	\item the aggregation algorithm;
	\item the coarse-space correction at the coarsest level.
\end{enumerate}
A list of the parameters that can be set, along with their allowed and
default values, is given in Tables~\ref{tab:p_type}-\ref{tab:p_coarse}.
For a detailed description  of the meaning of the parameters, please
refer to Section~\ref{sec:background}. 
%
%Note that the routine allows to set different features of the
%preconditioner at each level through the use of \verb|ilev|.
%This should be done by users with experience in the field of
%multi-level preconditioners. Non-expert users are recommended
%to call \verb| mld_precset| without specifying \verb|ilev|.

\bsideways
\begin{center}
\begin{tabular}{|l|l|p{2cm}|l|p{7cm}|}
\hline
\verb|what|              & \textsc{data type}        &  \verb|val|      &  \textsc{default}  &
\textsc{comments} \\ \hline
%\multicolumn{5}{|c|}{\emph{type of the multi-level preconditioner}}\\ \hline
\verb|mld_ml_type_|      & \verb|character(len=*)|
                         & \texttt{'ADD'} \ \ \ \texttt{'MULT'}   
                         & \texttt{'MULT'}
                         & Basic multi-level framework: additive or multiplicative
                           among the levels (always additive inside a level).         \\ \hline 
\verb|mld_smoother_type_|& \verb|character(len=*)|
                         & \texttt{'DIAG'} \ \ \ \texttt{'BJAC'} \ \ \ \texttt{'AS'}
                         & \texttt{'AS'}
                         & Basic one-level preconditioner (i.e.\ smoother): diagonal,
                           block Jacobi, AS. \\ \hline
\verb|mld_smoother_pos_| & \verb|character(len=*)|
                         & \texttt{'PRE'} \ \ \ \texttt{'POST'} \ \ \ \texttt{'TWOSIDE'}
                         & \texttt{'POST'}
                         & ``Position'' of the smoother: pre-smoother, post-smoother, 
                           pre- and post-smoother. \\
\hline
\end{tabular}
\end{center}
\caption{Parameters defining the type of multi-level preconditioner.
\label{tab:p_type}}                       
\esideways
                   
\bsideways
\begin{center}
\begin{tabular}{|l|l|p{3.2cm}|l|p{7cm}|}
\hline
\verb|what|              & \textsc{data type}        &  \verb|val|      &  \textsc{default}  &
\textsc{comments} \\ \hline
%\multicolumn{5}{|c|}{\emph{basic one-level preconditioner (smoother)}} \\ \hline
\verb|mld_sub_ovr_|       & \verb|integer|
                         & any~int.~num.~$\ge 0$
                         & 1
                         & Number of overlap layers. \\ \hline
\verb|mld_sub_restr_|    & \verb|character(len=*)|
                         & \texttt{'HALO'} \hspace{2.5cm} \texttt{'NONE'}
                         & \texttt{'HALO'}
                         & Type of restriction operator:
                           \texttt{'HALO'} for taking into account the overlap, \texttt{'NONE'} 
                           for neglecting it. \\ \hline
\verb|mld_sub_prol_|     & \verb|character(len=*)|
                         & \texttt{'SUM'} \hspace{2.5cm} \texttt{'NONE'}
                         & \texttt{'NONE'}
                         & Type of prolongator operator:
                           \texttt{'SUM'} for adding the contributions from the overlap, \texttt{'NONE'}
                           for neglecting them.   \\ \hline
\verb|mld_sub_solve_|    & \verb|character(len=*)|
                         & \texttt{'ILU'} \hspace{2.5cm} \texttt{'MILU'} \hspace{2.5cm} \texttt{'ILUT'} 
                           \hspace{2.5cm} \texttt{'UMF'} \hspace{2.5cm} \texttt{'SLU'}
                         & \texttt{'UMF'}
                         & Local solver: ILU($p$), MILU($p$), ILU($p,t$), LU from UMFPACK, LU from SuperLU
                           (plus triangular solve). \\ \hline  
\verb|mld_sub_fillin_|   & \verb|integer|
                         & Any~int.~num.~$\ge 0$
                         & 0
                         & Fill-in level $p$ of the incomplete LU factorizations. \\ \hline
\verb|mld_sub_iluthrs_|  & \verb|real(|\emph{kind\_parameter}\verb|)|
                         & Any~real~num.~$\ge 0$
                         & 0
                         & Drop tolerance $t$ in the ILU($p,t$) factorization. \\ \hline
\verb|mld_sub_ren_|      & \verb|character(len=*)|
                         & \texttt{'RENUM\_NONE'}  \texttt{'RENUM\_GLOBAL'} %, \texttt{'RENUM_GPS'}
                         & \texttt{'RENUM\_NONE'}
                         & Row and column reordering of the local submatrices: no reordering,
                           reordering according to the global numbering of the rows and columns of
                           the whole matrix. \\
\hline
\end{tabular}
\end{center}
\caption{Parameters defining the one-level preconditioner used as smoother.
\label{tab:p_smoother}}  
\esideways
                   
\bsideways
\begin{center}
\begin{tabular}{|l|l|p{2.3cm}|p{2.6cm}|p{7cm}|}
\hline
\verb|what|              & \textsc{data type}        &  \verb|val|      &  \textsc{default}  &
\textsc{comments} \\ \hline
%\multicolumn{5}{|c|}{\emph{aggregation algorithm}} \\ \hline
\verb|mld_aggr_alg_|     & \verb|character(len=*)|
                         & \texttt{'DEC'}
                         & \texttt{'DEC'}
                         & Aggregation algorithm. Currently, only the decoupled aggregation is available. \\ \hline
\verb|mld_aggr_kind_|    & \verb|character(len=*)|
                         & \texttt{'SMOOTH'} \hspace{2.5cm} \texttt{'RAW'}
                         & \texttt{'SMOOTH'}
                         & Type of aggregation: smoothed, raw (i.e.\ using the tentative prolongator). \\ \hline
\verb|mld_aggr_thresh_|  & \verb|real(|\emph{kind\_parameter}\verb|)|
                         & Any~real~num. $\in [0, 1]$
                         & 0
                         & The threshold $\theta$ in the aggregation algorithm. \\ \hline
\verb|mld_aggr_eig_|     & \verb|character(len=*)|
                         & \texttt{'A\_NORMI'}
                         & \texttt{'A\_NORMI'}
                         & Estimate of the eigenvalue $D^{-1}A$ with largest modulus,
                           to build the damping parameter $\omega$ in the smoothed aggregation.
                           Currently, only the infinity norm of
                           the matrix is available. \\ \hline
\verb|mld_aggr_damp_|    & \verb|real(|\emph{kind\_parameter}\verb|)|
                         & Any~real~num.
                         & $4/(3||D^{-1}A||_\infty)$
                         & The damping parameter $\omega$ in the smoothed aggregation algorithm. 
                           If the user specifies a negative value, then $\omega$ is set to its default
                           value; otherwise, $\omega$ is set to the value provided by the
                           user. In the latter case no estimate of the eigenvalue $D^{-1}A$ with
                           largest modulus is computed.\\
\hline
\end{tabular}
\end{center}
\caption{Parameters defining the aggregation algorithm.
\label{tab:p_aggregation}} 
\esideways
                     
\bsideways
\begin{center}
\begin{tabular}{|l|l|p{3.2cm}|l|p{7cm}|}
\hline
\verb|what|              & \textsc{data type}        &  \verb|val|      &  \textsc{default}  &
\textsc{comments} \\ \hline
%\multicolumn{5}{|c|}{\emph{coarse-space correction at the coarsest level}}\\ \hline
\verb|mld_coarse_solve_| & \verb|character(len=*)|
                         & \texttt{'BJAC'} \hspace{2.5cm} \texttt{'UMF'} \hspace{2.5cm} \texttt{'SLU'}
                           \hspace{2.5cm} \texttt{'SLUDIST'}
                         & \texttt{'BJAC'}
                         & Solver used at the coarsest level: block Jacobi, sequential LU from UMFPACK,
                           sequential LU from SuperLU, distributed LU from SuperLU\_Dist.
                           With \texttt{'SLUDIST'} the coarsest matrix
                           must be distributed; with \texttt{'UMF'} or
                           \texttt{'SLU'} it must be replicated. \\ \hline
\verb|mld_coarse_mat_|   & \verb|character(len=*)|
                         & \texttt{'DISTR'} \hspace{2.5cm} \texttt{'REPL'}
                         & \texttt{'DISTR'}
                         & Coarsest matrix: distributed among the processors or replicated on each of them. \\ \hline
\verb|mld_coarse_subsolve_| & \verb|character(len=*)|
                         & \texttt{'ILU'} \hspace{2.5cm} \texttt{'MILU'} \hspace{2.5cm} \texttt{'ILUT'}
                           \hspace{2.5cm} \texttt{'UMF'} \hspace{2.5cm} \texttt{'SLU'}
                         & \texttt{'UMF'}
                         & Solver for the diagonal blocks of the coarse matrix, in case the block Jacobi solver
                           is chosen as coarsest-level solver: ILU($p$), MILU($p$), ILU($p,t$), LU from UMFPACK,
                           LU from SuperLU, plus triangular solve. \\ \hline
\verb|mld_coarse_sweeps_|& \verb|integer|                         
                         & Any~int.~num.~$> 0$
                         & 4
                         & Number of Block-Jacobi sweeps when 'BJAC' is used as coarsest-level solver. \\ \hline
\verb|mld_coarse_fillin_| & \verb|integer|
                         & Any~int.~num.~$\ge 0$
                         & 0
                         & Fill-in level $p$ of the incomplete LU factorizations. \\ \hline
\verb|mld_coarse_iluthrs_| & \verb|real(|\emph{kind\_parameter}\verb|)|
                         & Any~real.~num.~$\ge 0$
                         & 0
                         & Drop tolerance $t$ in the ILU($p,t$) factorization. \\
\hline
\end{tabular}
\end{center}
\caption{Parameters defining the coarse-space correction at the coarsest
level.\label{tab:p_coarse}} 
\esideways


\clearpage

\subsection{Subroutine mld\_precbld\label{sec:precbld}}

\begin{center}
\verb|mld_precbld(a,desc_a,p,info)|\\
\end{center}

\noindent
This routine builds the preconditioner according to the requirements made by
the user through the routines \verb|mld_precinit| and \verb|mld_precset|.

{\vskip2\baselineskip\noindent\large\bfseries Arguments}

\begin{tabular}{p{1.2cm}p{12cm}}
\verb|a|      & \verb|type(psb_|\emph{x}\verb|spmat_type), intent(in)|. \\
              & The sparse matrix structure containing the local part of the
                matrix to be preconditioned. Note that \emph{x} must be chosen according
                to the real/complex, 
single/double precision version of MLD2P4 under use.
                See the PSBLAS User's Guide for details \cite{PSBLASGUIDE}.\\
\verb|desc_a| & \verb|type(psb_desc_type), intent(in)|. \\
              & The communication descriptor of \verb|a|. See the PSBLAS User's Guide for
                details \cite{PSBLASGUIDE}.\\
\verb|p|      & \verb|type(mld_|\emph{x}\verb|prec_type), intent(inout)|.\\
              & The preconditioner data structure. Note that \emph{x} must be chosen according
                to the real/complex, single/double precision version of MLD2P4 under use.\\
\verb|info|   & \verb|integer, intent(out)|.\\
              & Error code. If no error, 0 is returned. See Section~\ref{sec:errors} for details.\\
\end{tabular}

\clearpage
\subsection{Subroutine mld\_precaply\label{sec:precaply}}

\begin{center}
\verb|mld_precaply(p,x,y,desc_a,info)|\\
\verb|mld_precaply(p,x,y,desc_a,info,trans,work)|\\
\end{center}

\noindent
This routine computes $y = op(M^{-1})\, x$, where $M$ is a previously built
preconditioner, stored into \verb|p|, and $op$
denotes the preconditioner itself or its transpose, according to
the value of \verb|trans|.
Note that, when MLD2P4 is used with a Krylov solver from PSBLAS,
\verb|mld_precaply| is called within the PSBLAS routine \verb|mld_krylov|
and hence it is completely transparent to the user.

{\vskip2\baselineskip\noindent\large\bfseries Arguments}

\begin{tabular}{p{1.2cm}p{12cm}}
\verb|p|      & \verb|type(mld_|\emph{x}\verb|prec_type), intent(inout)|.\\
              & The preconditioner data structure, containing the local part of $M$.
                Note that \emph{x} must be chosen according
                to the real/complex, single/double precision version of MLD2P4 under use.\\
\verb|x|      & \emph{type}\verb|(|\emph{kind\_parameter}\verb|), dimension(:), intent(in)|.\\
              & The local part of the vector $x$. Note that \emph{type} and   
                \emph{kind\_parameter} must be chosen according
                to the real/complex, single/double precision version of MLD2P4 under use.\\
\verb|y|      & \emph{type}\verb|(|\emph{kind\_parameter}\verb|), dimension(:), intent(out)|.\\
              & The local part of the vector $y$. Note that \emph{type} and
                \emph{kind\_parameter} must be chosen according
                to the real/complex, single/double precision version of MLD2P4 under use.\\
\verb|desc_a| & \verb|type(psb_desc_type), intent(in)|. \\
              & The communication descriptor associated to the matrix to be
                preconditioned.\\
\verb|info|   & \verb|integer, intent(out)|.\\
              & Error code. If no error, 0 is returned. See Section~\ref{sec:errors} for details.\\
\verb|trans|  & \verb|character(len=1), optional, intent(in).|\\
              & If \verb|trans| = \verb|'N','n'| then $op(M^{-1}) = M^{-1}$;
                if \verb|trans| = \verb|'T','t'| then $op(M^{-1}) = M^{-T}$
                (transpose of $M^{-1})$;  if \verb|trans| = \verb|'C','c'| then $op(M^{-1}) = M^{-C}$
                (conjugate transpose of $M^{-1})$.\\
\verb|work|  & \emph{type}\verb|(|\emph{kind\_parameter}\verb|), dimension(:), optional, target|.\\
             & Workspace. Its size should be at
               least \verb|4 * psb_cd_get_local_| \verb|cols(desc_a)| (see the PSBLAS User's Guide).
               Note that \emph{type} and \emph{kind\_parameter} must be chosen according
               to the real/complex, single/double precision version of MLD2P4 under use.\\
\end{tabular}

\clearpage

\subsection{Subroutine mld\_precfree\label{sec:precfree}}

\begin{center}
\verb|mld_precfree(p,info)|\\
\end{center}

\noindent
This routine deallocates the preconditioner data structure.

{\vskip2\baselineskip\noindent\large\bfseries Arguments}

\begin{tabular}{p{1.2cm}p{10.5cm}}
\verb|p|      & \verb|type(mld_|\emph{x}\verb|prec_type), intent(inout)|.\\
              & The preconditioner data structure. Note that \emph{x} must be chosen according
                to the real/complex, single/double precision version of MLD2P4 under use.\\
\verb|info|   & \verb|integer, intent(out)|.\\
              & Error code. If no error, 0 is returned. See Section~\ref{sec:errors} for details.\\
\end{tabular}

\clearpage

\subsection{Subroutine mld\_precdescr\label{sec:precdescr}}

\begin{center}
\verb|mld_precdescr(p,info)|\\
\verb|mld_precdescr(p,info,iout)|\\
\end{center}

\noindent
This routine prints a description of the preconditioner to the standard output or
to a file. It must be called after \verb|mld_precbld| has been called.

{\vskip2\baselineskip\noindent\large\bfseries Arguments}

\begin{tabular}{p{1.2cm}p{12cm}}
\verb|p|      & \verb|type(mld_|\emph{x}\verb|prec_type), intent(in)|.\\
              & The preconditioner data structure. Note that \emph{x} must be chosen according
                to the real/complex, single/double precision version of MLD2P4 under use.\\
\verb|info|   & \verb|integer, intent(out)|.\\
              & Error code. If no error, 0 is returned. See Section~\ref{sec:errors} for details.\\
\verb|iout|   & \verb|integer, intent(in), optional|.\\
              & The id of the file where the preconditioner description
                will be printed; the default is the standard output.\\
\end{tabular}

%%% Local Variables: 
%%% mode: latex
%%% TeX-master: "userguide"
%%% End: 
