\section{Data Structures}
\label{sec:datastruct}
%\ifthenelse{\boolean{mtc}}{\minitoc}{}

In this chapter are illustrated data structures used for definition of
routines interfaces. This include data structure for sparse matrix,
communication descriptor and preconditioner. These data structures are used for
calling PSBLAS routines in Fortran~90 language and will be used to next
chapters containing these callings. Their definitions are included in
the modules \verb|psb_spmat_type|, \verb|psb_descriptor_type| and \verb|psb_prec_type|. 

\subsection{Sparse Matrix data structure}
\label{sec:spmat}
The \hypertarget{spdata}{{\tt psb\_spmat\_type}} data structure
contains all information about local portion of the sparse matrix and   
its storage mode. Many of this fields are set in fully-transparent
mode by PSBLAS-TOOLS routines when inserting a new sparse matrix, user
must set only fields which describe matrix storage mode. \\
Fields contained in Sparse matrix structures are:
\begin{description}
\item[{\bf aspk}] Contains values of the local distributed sparse
matrix.\\
Specified as: a pointer to an array of rank one of type corresponding
to matrix entries type .
\item[{\bf ia1}] Holds integer information on distributed sparse
matrix. Actual information will depend on data format used.\\
Specified as: a pointer to an integer array of rank one.
\item[{\bf ia2}] Holds integer information on distributed sparse
matrix. Actual information will depend on data format used.\\
Specified as: a pointer to an integer array of rank one.
\item[{\bf infoa}] On entry can hold auxiliary information on distributed sparse
matrix.  Actual information will depend on data format used.\\
Specified as: integer array of length 10.
\item[{\bf fida}] Defines the format of the distributed sparse matrix.\\
Specified as: a string of length 5
\item[{\bf descra}] Describe the characteristic of the distributed sparse matrix.\\
Specified as: array of character of length 9.
\item[{\bf pl}] Specifies the local row permutation of distributed sparse
matrix. If pl(1) is equal to 0, then there isn't row permutation.\\
Specified as: pointer to integer array of dimension equal to number of local row (matrix\_data[psb\_n\_row\_\hbox{]})
\item[{\bf pr}] Specifies the local column permutation of distributed sparse
matrix. If PR(1) is equal to 0, then there isn't columnm permutation.\\
Specified as: pointer to integer array of dimension equal to number of
local row (matrix\_data[psb\_n\_col\_\hbox{]})
\item[{\bf m}] Number of rows; if row indices are stored explicitly,
as in Coordinate Storage, should be greater than or equal to the
maximum row index actually present in the sparse matrix.
Specified as: integer variable.
\item[{\bf k}] Number of columns; if column indices are stored explicitly,
as in Coordinate Storage or Compressed Sparse Rows, should be greater
than or equal to the maximum column  index actually present in the sparse matrix.
Specified as: integer variable.
\end{description}
FORTRAN95 interface for distributed sparse matrices containing double precision
real entries is defined as in figure~\ref{fig:spmattype}.
\begin{figure}[h!]
  \begin{Sbox}
    \begin{minipage}[tl]{0.85\textwidth}
\begin{verbatim}
type psb_dspmat_type
   integer     :: m, k
   character   :: fida(5)
   character   :: descra(10)
   integer     :: infoa(10)
   real(kind(1.d0)), pointer :: aspk(:)
   integer, pointer :: ia1(:), ia2(:), pr(:), pl(:)
end type psb_dspmat_type
\end{verbatim}
    \end{minipage}
  \end{Sbox}
  \setlength{\fboxsep}{8pt}
  \begin{center}
    \fbox{\TheSbox}
  \end{center}
  \caption{\label{fig:spmattype}The PSBLAS defined data type that contains a sparse matrix.}
\end{figure}

The following two cases are among the most commonly used: 
\begin{description}
\item[fida=``CSR''] Compressed storage by rows. In this case the
following should hold: 
\begin{enumerate}
\item \verb|ia2(i)| contains the index of the first element of row
\verb|i|; the last element of the sparse matrix is thus stored at
index $ia2(m+1)-1$. It should contain \verb|m+1| entries in
nondecreasing order (strictly increasing, if there are no empty rows).
\item \verb|ia1(j)| contains the column index and \verb|aspk(j)|
contains the corresponding coefficient value, for all $ia2(1) \le j
\le ia2(m+1)-1$.
\end{enumerate}
\item[fida=``COO''] Coordinate storage. In this case the following
should hold: 
\begin{enumerate}
\item \verb|infoa(1)| contains the number of nonzero elements in the
matrix; 
\item For all $1 \le j \le infoa(1)$, the coefficient, row index and
column index are stored into \verb|apsk(j)|, \verb|ia1(j)| and
\verb|ia2(j)| respectively. 
\end{enumerate}
\end{description}


\subsection{Descriptor data structure}
\label{sec:desc}
All the general matrix informations and elements to be
exchanged among processes are stored within a data structure of the
type \hypertarget{descdata}{{\tt psb\_desc\_type}}. 
Every structure of this type is associated to a sparse matrix, it
contains data about general matrix informations and elements to be
exchanged among processes.  \\ 
It is not necessary for the user to
know the internal structure of \verb|psb_desc_type|, it is set in
fully-transparent mode by PSBLAS-TOOLS routines when inserting a new
sparse matrix, however the definition of the descriptor is the
following.  
\begin{description}
\item[{\bf matrix\_data}] includes general information about matrix and
BLACS grid. More precisely:
\begin{description}
\item[matrix\_data[psb\_dec\_type\_\hbox{]}] Identifies the decomposition type
(global); the actual values are internally defined, so they should
never be accessed directly.
\item[matrix\_data[psb\_ctxt\_\hbox{]}] Communication context as returned by the
BLACS (global).
\item[matrix\_data[psb\_m\_\hbox{]}] Total number of equations (global).
\item[matrix\_data[psb\_n\_\hbox{]}] Total number of variables (global).
\item[matrix\_data[psb\_n\_row\_\hbox{]}] Number of grid variables owned by the
current process (local); equivalent to the number of local rows in the
sparse coefficient matrix.
\item[matrix\_data[psb\_n\_col\_\hbox{]}] Total number of grid variables read by the
current process (local); equivalent to the number of local columns in
the sparse coefficient matrix. They include the halo.
\end{description}
Specified as: a pointer to integer array of dimension 10.
\item[{\bf halo\_index}] A list of the halo and boundary elements for
the current process to be exchanged with other processes; for each
processes with which it is necessary to communicate:
\begin{enumerate}
\item Process identifier;
\item Number of points to be received;
\item Indices of points to be received;
\item Number of points to be sent;
\item Indices of points to be sent;
\end{enumerate}
The list may contain an arbitrary number of groups; its end is marked
by a -1.\\
Specified as: a pointer to an integer array of rank one.
\item [{\bf ovrlap\_index}] A list of the overlap elements for the
current process, organized in groups like the previous vector:
\begin{enumerate}
\item Process identifier;
\item Number of points to be received;
\item Indices of points to be received;
\item Number of points to be sent;
\item Indices of points to be sent;
\end{enumerate}
The list may contain an arbitrary number of groups; its end is marked
by a -1.\\
Specified as: a pointer to an integer array  of rank one.
\item [{\bf ovrlap\_index}] For all overlap points belonging to th
ecurrent process:
\begin{enumerate}
\item  Overlap point index;
\item  Number of processes sharing that overlap points;
\end{enumerate}
The list may contain an arbitrary number of groups; its end is marked
by a -1.\\
Specified as: a pointer to an integer array of rank one.
\item[{\bf loc\_to\_glob}] each element $i$ of this array contains
global identifier of the local variable $i$.\\
Specified as: a pointer to an integer array of rank one.
\item[{\bf glob\_to\_loc}]  if global variable $i$ is read by current
process then element $i$ contains local index correpondent to global variable $i$;
else element $i$ contains -1 (NULL) value.\\
Specified as: a pointer to an integer array of rank one.
\end{description}
FORTRAN95 interface for \verb|psb_desc_type| structures is therefore defined
as follows:
\begin{figure}[h!]
  \begin{Sbox}
    \begin{minipage}[tl]{0.9\textwidth}
\begin{verbatim} 
type psb_desc_type 
   integer, pointer :: matrix_data(:), halo_index(:)
   integer, pointer :: overlap_elem(:), overlap_index(:)
   integer, pointer :: loc_to_glob(:), glob_to_loc(:)
end type psb_desc_type 
\end{verbatim}
    \end{minipage}
  \end{Sbox}
  \setlength{\fboxsep}{8pt}
  \begin{center}
    \fbox{\TheSbox}
  \end{center}
  \caption{\label{fig:desctype}The PSBLAS defined data type that
    contains the communication descriptor.}
\end{figure}

\subsection{Preconditioner data structure}
\label{sec:prec}
PSBLAS-2.0 offers the possibility to use many different types of
preconditioning schemes. Besides the simple well known preconditioners
like Diagonal Scaling or Block Jacobi (with ILU(0) incomplete
factorization) also more complex preconditioning methods are
implemented like the Additive Schwarz and Two-Level ones. A
preconditioner is held in the \hypertarget{precdata}{{\tt psb\_prec\_type}} data structure
which depends on the \verb|psb_base_prec| reported in
figure~\ref{fig:prectype}. The \verb|psb_base_prec| 
data type may contain a simple preconditioning matrix with the
associated communication descriptor which may be different than the
system communication descriptor in the case of parallel
preconditioners like the Additive Schwarz one. Then the
\verb|psb_prec_type| may contain more than one preconditioning matrix
like in the case of Two-Level (in general Multi-Level) preconditioners.
The user can choose the type of preconditioner to be used by means of
the \verb|psb_precset| subroutine; once the type of preconditioning
method is specified, along with all the parameters that characterize
it, the preconditioner data structure can be built using the
\verb|psb_precbuild| subroutine.
This data structure wants to be flexible enough to easily allow the
implementation of new kind of preconditioners. The values contained in
the \verb|iprcparm| and \verb|dprcparm| define tha type of
preconditioner along with all the parameters related to it; thus,
\verb|iprcparm| and \verb|dprcparm| define how the other records have
to be interpreted.
\begin{figure}[h!]
  \small
  \begin{Sbox}
    \begin{minipage}[tl]{0.9\textwidth}
\begin{verbatim}
  type psb_base_prec

    type(psb_spmat_type), pointer  :: av(:) => null()
    real(kind(1.d0)), pointer      :: d(:)  => null()
    type(psb_desc_type), pointer   :: desc_data => null()
    integer, pointer               :: iprcparm(:) => null()
    real(kind(1.d0)), pointer      :: dprcparm(:) => null()
    integer, pointer               :: perm(:)  => null()
    integer, pointer               :: mlia(:)  => null()
    integer, pointer               :: invperm(:) => null()
    integer, pointer               :: nlaggr(:)  => null()
    type(psb_spmat_type), pointer  :: aorig    => null()
    real(kind(1.d0)), pointer      :: dorig(:) => null()
    
 end type psb_base_prec
  
  type psb_prec_type
    type(psb_base_prec), pointer  :: baseprecv(:) => null()
    integer                       :: prec, base_prec
 end type psb_prec_type
\end{verbatim}
    \end{minipage}
  \end{Sbox}
  \setlength{\fboxsep}{8pt}
  \begin{center}
    \fbox{\TheSbox}
  \end{center}
  \caption{\label{fig:prectype}The PSBLAS defined data type that contains a preconditioner.}
\end{figure}




%%% Local Variables: 
%%% mode: latex
%%% TeX-master: "userguide"
%%% End: 
