\section{Preconditioner routines}
\label{sec:precs}

% \section{Preconditioners}
\label{sec:psprecs}
PSBLAS contains the implementation of many preconditioning
techniques some of which are very flexible thanks to the presence of
many parameters that is possible to adjust to fit the user's needs:
\begin{itemize}
\item Diagonal Scaling
\item Block Jacobi with ILU(0) factorization
%% \item Additive Schwarz with the Restricted Additive Schwarz and
%%   Additive Schwarz with Harmonic extensions;
\end{itemize}
The PSBLAS library is incorporating a package of two-level Additive
  Schwarz preconditioners called MD2P4; this is actually a family of 
  preconditioners since there is the possibility to choose between
  many variants, and is currently in an experimental state. Its
  documentation is planned to appear after stabilization of the
  package, which will characterize release  2.1 of our library.




\subroutine{psb\_precset}{Sets the preconditioner type}

\syntax{call psb\_precset}{prec, ptype, iv, rs, ierr}

\begin{description}
\item[\bf On Entry]
\item[prec] 
Scope: {\bf local} \\
Type: {\bf required}\\
Specified as: a pronditioner data structure \precdata.
\item[ptype] the type of preconditioner. 
Scope: {\bf global} \\
Type: {\bf required}\\
Specified as: a character string, see usage notes.
\item[iv] integer parameters for the precondtioner. 
Scope: {\bf global} \\
Type: {\bf required}\\
Specified as: an integer array, see usage notes. 
\item[rs] 
Scope: {\bf global} \\
Type: {\bf optional}\\
Specified as: a long precision real number.

\item[ierr] 
Scope: {\bf global} \\
Type: {\bf required}\\
\end{description}
\section*{Usage Notes}
The PSBLAS 2.0 contains a number of preconditioners, ranging from a
simple diagonal scaling to 2-level domain decomposition. These
preconditioners may use the SuperLU or the UMFPACK software, if
installed; see~\cite{SUPERLU,UMFPACK}. 
Legal inputs to this subroutine are interpreted depending on the
$ptype$ string as follows\footnote{The string is case-insensitive}:
\begin{description}
\item[NONE] No preconditioning, i.e. the preconditioner is just a copy
  operator.
\item[DIAG] Diagonal scaling; each entry of the input vector is
  multiplied by the reciprocal of the sum of the absolute values of
  the coefficients in the corresponding row of matrix  $A$;
\item[ILU] Precondition by the incomplete LU factorization of the
  block-diagonal of matrix $A$, where block boundaries are determined
  by the data allocation boundaries for each process; requires no
  communication. Only $ILU(0)$ is currently implemented. 
%% \item[AS] Additive Schwarz preconditioner (see~\cite{PARA04}); in this
%%   case the user may specify additional flags through the integer
%%   vector \verb|ir| as follows:
%% \begin{description}
%% \item[$iv(1)$] Number of overlap levels, an integer $novr>=0$, default
%%   $novr=1$.  
%% \item[$iv(2)$] Restriction operator, legal values: \verb|psb_halo_|,
%%   \verb|psb_none_|; default: \verb|psb_halo_|
%% \item[$iv(3)$] Prolongation operator, legal values: \verb|psb_none_|,
%%   \verb|psb_sum_|, \verb|psb_avg_|, default: \verb|psb_none_|
%% \item[$iv(4)$] Factorization type, legal values: \verb|f_ilu_n_|,
%%   \verb|f_slu_|, \verb|f_umf_|, default: \verb|f_ilu_n_|.
%% \end{description}
%% Note that the default corresponds to a Restricted Additive Schwarz
%% preconditioner with $ILU(0)$ and 1 level of overlap.
%% \item[2L] Second level of a multilevel preconditioner, see below
%% \end{description}
%% If a multilevel preconditioner is desired, the user should call
%% \verb|psb_precset| twice, the first time  choosing an AS variant, and
%% a second time specifying 
%% $ptype=2L$ with the following optional parameters in $iv$ (see
%% also~\cite{APNUM06,DD2}): 
%% \begin{description}
%% \item[$iv(1)$] Type of multilevel correction, legal values: \verb|no_ml_|,
%%   \verb|add_ml_prec_|, \verb|mult_ml_prec_|,
%%   default: \verb|mult_ml_prec_|;
%% \item[$iv(2)$] Aggregation algorithm, legal values: \verb|loc_aggr_|;
%% \item[$iv(3)$] Smoother type, legal values: \verb|no_smth_|,
%%   \verb|smth_omg_|, default: \verb|smth_omg_|;
%% \item[$iv(4)$] Coarse matrix allocation, legal values:
%%   \verb|mat_distr_|, \verb|mat_repl_|, default: \verb|mat_distr_|
%% \item[$iv(5)$] Smoother position, legal values: \verb|pre_smooth_|,
%%   \verb|post_smooth_|, \verb|smooth_both_|, default:
%%   \verb|post_smooth_|
%% \item[$iv(6)$] Factorization type (for coarse matrix), legal values: \verb|f_ilu_n_|,
%%   \verb|f_slu_|, \verb|f_umf_|, default: \verb|f_ilu_n_|;
%% \item[$iv(7)$] Number of Jacobi sweeps for coarse system correction,
%%   default 1.
%% \item[$rs$] Set the smoother parameter $\omega$ a user defined value;
%%   default: esitimate with the infinity norm of matrix $A$.
\end{description}
%% The 2-level preconditioners are based on the idea of building a
%% coarse-space approximation $A_C$ of the matrix $A$; given a set $W_C$
%% of coarse vertices, with size $n_C$, and a suitable restriction
%% operator $R_C \in \Re^{n_C \times n}$, $A_C$ is defined as 
%% \[
%% A_C=R_C A R_C^T .
%% \]
%% The prolongator $R_C^T$ is built with the smoothed aggregation technique,
%% in which we start from a tentative prolongator that simply maps
%% fine-level entries onto their aggregates $P_C$; if the user chooses
%% \verb|no_smth_| this is the prolongator used, otherwise it is
%% multiplied by a smoother \[ S = I - \omega D^{-1} A \], where $D$ is
%% the diagonal of $A$ and  $\omega$ may be imposed by the user or
%% estimated internally.   
%% The coarse space correction may be added to the fine level solution
%% \verb|add_ml_prec_|
%% \[
%% M_{2L-A}^{-1} = M_{C}^{-1} + M_{1L}^{-1}. 
%% \]
%% or it can be composed  in a multiplicative framework
%% (\verb|mult_ml_prec_|)as a pre-smoothed correction (\verb|pre_smooth_|)
%% \[
%% M_{2L-H1}^{-1} = M_{C}^{-1} + \left( I - M_{C}^{-1}A \right) M_{1L}^{-1}, 
%% \]
%% post-smoothed correction (\verb|post_smooth_|)
%% \[
%% M_{2L-H2}^{-1} = M_{1L}^{-1} + \left( I - M_{1L}^{-1}A \right) M_{C}^{-1}. 
%% \]
%% or two-sided for symmetric matrices (\verb|smooth_both_|).



\subroutine{psb\_precbld}{Builds a preconditioner}

\syntax{call psb\_precbld}{a, desc\_a, prec, info, upd}

\begin{description}
\item[\bf On Entry]
\item[a] the system sparse matrix.
Scope: {\bf local} \\
Type: {\bf required}\\
Specified as: a sparse matrix data structure \spdata.
\item[desc\_a] the problem communication descriptor. 
Scope: {\bf local} \\
Type: {\bf required}\\
Specified as: a communication descriptor data structure \descdata.
\item[upd] 
Scope: {\bf global} \\
Type: {\bf optional}\\
Specified as: a character.
\end{description}

\begin{description}
\item[\bf On Return]
\item[prec] the preconditioner.\\
Scope: {\bf local} \\
Type: {\bf required}\\
Specified as: a precondtioner data structure \precdata\\
\item[info] the return error code.\\
Scope: {\bf global} \\
Type: {\bf required}\\
Specified as: an integer, upon successful completion $info=0$ \\
\end{description}



\subroutine{psb\_precaply}{Preconditioner application routine}

\syntax{call psb\_precaply}{prec,x,y,desc\_a,info,trans,work}
\syntax{call psb\_precaply}{prec,x,desc\_a,info,trans}

\begin{description}
\item[\bf On Entry]
\item[prec] the preconditioner.
Scope: {\bf local} \\
Type: {\bf required}\\
Specified as: a preconditioner data structure \precdata.
\item[x] the source vector.
Scope: {\bf local} \\
Type: {\bf require}\\
Specified as: a double precision array.
\item[desc\_a] the problem communication descriptor.
Scope: {\bf local} \\
Type: {\bf required}\\
Specified as: a communication data structure \descdata.
\item[trans] 
Scope: {\bf } \\
Type: {\bf optional}\\
Specified as: a character.
\item[work] an optional work space
Scope: {\bf local} \\
Type: {\bf optional}\\
Specified as: a double precision array.
\end{description}

\begin{description}
\item[\bf On Return]
\item[y] the destination vector.
Scope: {\bf local} \\
Type: {\bf required}\\
Specified as: a double precision array.
\item[info] the return error code.\\
Scope: {\bf local} \\
Type: {\bf required}\\
Specified as: an integer, upon successful completion $info=0$ .\\
\end{description}



\subroutine{psb\_prec\_descr}{Prints a description of current preconditioner}

\syntax{call psb\_prec\_descr}{prec}

\begin{description}
\item[\bf On Entry]
\item[prec] the preconditioner.
Scope: {\bf local} \\
Type: {\bf required}\\
Specified as: a preconditioner data structure \precdata.
\end{description}



%%% Local Variables: 
%%% mode: latex
%%% TeX-master: "userguide"
%%% End: 
