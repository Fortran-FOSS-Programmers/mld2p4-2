\section{Utilities}
\label{sec:util}


\subroutine{PSB\_HBIO\_MOD}{Input/Output in Harwell-Boeing format}

\subroutine*{hb\_read}{Read a sparse matrix from a file}
\syntax{call hb\_read}{a, iret, iunit, filename, b, mtitle}
 
\begin{description}
\item[\bf  On Entry ]
\item[filename] The name of the file to be read.\\
Type:{\bf optional}.\\
Specified as: a character variable containing a valid file name, or
\verb|-|, in which case the default input unit  5 (i.e. standard input
in Unix jargon) is used. Default: \verb|-|. 
\item[iunit] The Fortran file unit number.\\
Type:{\bf optional}.\\
Specified as: an integer value. Only meaningful if filename is not \verb|-|.
\end{description}

\begin{description}
\item[\bf On Return]
\item[a] the sparse matrix read from file.\\
Type:{\bf required}.\\
Specified as: a structured data of type \spdata.
\item[b] Rigth hand side.\\
Type: {\bf Optional} \\
An  array of type real or complex, rank 1 and having the ALLOCATABLE
attribute; will be allocated and filled in if the input file contains
a right hand side. 
\item[mtitle] Matrix title.\\
Type: {\bf Optional} \\
A charachter variable of length 72 holding a copy of the
matrix title as specified by the Harwell-Boeing format and contained
in the input file.
\item[iret] Error code.\\
Type: {\bf required} \\
An integer value; 0 means no error has been detected. 
\end{description}



\subroutine*{hb\_write}{Write a sparse matrix to a  file}
\syntax{call hb\_write}{a, iret, iunit, filename, key, rhs, mtitle}



\begin{description}
\item[\bf  On Entry ]
\item[a] the sparse matrix to be written.\\
Type:{\bf required}.\\
Specified as: a structured data of type \spdata.
\item[b] Rigth hand side.\\
Type: {\bf Optional} \\
An  array of type real or complex, rank 1 and having the ALLOCATABLE
attribute; will be allocated and filled in if the input file contains
a right hand side. 
\item[filename] The name of the file to be written to.\\
Type:{\bf optional}.\\
Specified as: a character variable containing a valid file name, or
\verb|-|, in which case the default output unit  6 (i.e. standard output
in Unix jargon) is used. Default: \verb|-|. 
\item[iunit] The Fortran file unit number.\\
Type:{\bf optional}.\\
Specified as: an integer value. Only meaningful if filename is not \verb|-|.
\item[key] Matrix key.\\
Type: {\bf Optional} \\
A charachter variable of length 8 holding the
matrix key as specified by the Harwell-Boeing format and to be
written to file.
\item[mtitle] Matrix title.\\
Type: {\bf Optional} \\
A charachter variable of length 72 holding the
matrix title as specified by the Harwell-Boeing format and to be
written to file.
\end{description}

\begin{description}
\item[\bf On Return]
\item[iret] Error code.\\
Type: {\bf required} \\
An integer value; 0 means no error has been detected. 
\end{description}




\subroutine{PSB\_MMIO\_MOD}{Input/Output in MatrixMarket format}

\subroutine*{mm\_mat\_read}{Read a sparse matrix from a file}
\syntax{call mm\_mat\_read}{a, iret, iunit, filename}

\begin{description}
\item[\bf  On Entry ]
\item[filename] The name of the file to be read.\\
Type:{\bf optional}.\\
Specified as: a character variable containing a valid file name, or
\verb|-|, in which case the default input unit  5 (i.e. standard input
in Unix jargon) is used. Default: \verb|-|. 
\item[iunit] The Fortran file unit number.\\
Type:{\bf optional}.\\
Specified as: an integer value. Only meaningful if filename is not \verb|-|.
\end{description}

\begin{description}
\item[\bf On Return]
\item[a] the sparse matrix read from file.\\
Type:{\bf required}.\\
Specified as: a structured data of type \spdata.
\item[iret] Error code.\\
Type: {\bf required} \\
An integer value; 0 means no error has been detected. 
\end{description}



\subroutine*{mm\_mat\_write}{Write a sparse matrix to a  file}
\syntax{call mm\_mat\_write}{a, mtitle, iret, iunit, filename}
\begin{description}
\item[\bf  On Entry ]
\item[a] the sparse matrix to be written.\\
Type:{\bf required}.\\
Specified as: a structured data of type \spdata.
\item[mtitle] Matrix title.\\
Type: {\bf required} \\
A charachter variable holding a descriptive title for the matrix to be
 written to file.
\item[filename] The name of the file to be written to.\\
Type:{\bf optional}.\\
Specified as: a character variable containing a valid file name, or
\verb|-|, in which case the default output unit  6 (i.e. standard output
in Unix jargon) is used. Default: \verb|-|. 
\item[iunit] The Fortran file unit number.\\
Type:{\bf optional}.\\
Specified as: an integer value. Only meaningful if filename is not \verb|-|.
\end{description}

\begin{description}
\item[\bf On Return]
\item[iret] Error code.\\
Type: {\bf required} \\
An integer value; 0 means no error has been detected. 
\end{description}



%%% Local Variables: 
%%% mode: latex
%%% TeX-master: "userguide"
%%% End: 
