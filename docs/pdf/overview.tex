\section{General Overview\label{sec:overview}}

The \emph{Multi-Level Domain Decomposition Parallel Preconditioners Package based on
PSBLAS (MLD2P4}) provides various versions of multi-level Schwarz preconditioners~\cite{DD2},
to be used in the iterative solutions of sparse linear systems $Ax=b$, where
$A$ is a square, real or complex, sparse matrix with a symmetric sparsity pattern.
\textbf{Ma non abbiamo detto che, se il pattern di sparista' non e' simmetrico,
lavoriamo su $(A+A^T)/2$? Ma questo vale solo per l'aggregazione? Dovremmo fare
qualcosa di consistente anche con 1-lev Schwarz.}
Both additive and hybrid preconditioners, i.e.\ multiplicative among the levels
and additive inside a level, are implemented; the basic additive Schwarz preconditioners
are obtained by considering only one level. A purely algebraic approach is used to
generate a sequence of coarse-level corrections to a basic preconditioner, without
explicitly using any information on the geometry of the original problem (e.g.\ the
discretization of a PDE). The smoothed aggregation technique is applied
as algebraic coarsening strategy~\cite{}.

The package is written in Fortran~95, using object-oriented techniques,
and is based on a distributed-memory parallel programming paradigm. \textbf{SALVATORE,
potresti aggiungere due righe sulla scelta del Fortran 95 e sul semplice interfacciamento
con i legacy codes, senza ripetere quello che e' detto sotto sulla scelta di PSBLAS?}
Single and double precision implementations of MLD2P4 are available for both the
real and the complex case, that can be used through a single interface.
\textbf{SALVATORE, funziona tutto?}

MLD2P4 has been designed to implement scalable and easy-to-use multilevel preconditioners
in the context of the PSBLAS (Parallel Sparse BLAS) computational framework~\cite{}.
PSBLAS is a library originally developed to address the parallel implementation of
iterative solvers for sparse linear system, by providing basic linear algebra
operators and data management facilities for distributed sparse matrices; it
also includes parallel Krylov solvers, built on the top of the basic PSBLAS kernels.
The preconditioners available in MLD2P4 can be used with these Krylov solvers.
The choice of PSBLAS has been mainly motivated by the need of having
a portable and efficient software infrastructure implementing ``de facto'' standard
parallel sparse linear algebra kernels, to pursue goals such as performance,
portability, modularity ed extensibility in the development of the preconditioner
package. On the other hand, the implementation of MLD2P4 has led to some
revisions and extentions of the PSBLAS kernels, leading to the
recent PSBLAS 2.0 version~\cite{}. The inter-process comunication required
by MLD2P4 is encapsulated into the PSBLAS routines, except few cases where
MPI~\cite{} is explicitly called. Therefore, MLD2P4 can be run on any parallel
machine where PSBLAS and MPI implementations are available.

MLD2P4 has a layered and modular software architecture where three main layers can be identified. The lower layer consists of the PSBLAS kernels, the middle one implements
the construction and application phases of the preconditioners, and the upper one
provides a uniform and easy-to-use interface to all the preconditioners. 
This architecture allows for different levels of use of the package:
few black-box routines at the upper level allow non-expert users to easily
build any preconditioner available in MLD2P4 and to apply it within a PSBLAS Krylov solver.
On the other hand, the routines of the middle and lower layer can be used and extended
by expert users to build new versions of multi-level Schwarz preconditioners.\\

\textbf{Organizzazione della guida:\\
dire che per il momento non
forniamo anche la documentazione del middle layer, ma lo faremo in seguito\\}

\textbf{Evidenziare le parole chiave che caratterizzano il nostro package}

%%% Local Variables: 
%%% mode: latex
%%% TeX-master: "userguide"
%%% End: 
