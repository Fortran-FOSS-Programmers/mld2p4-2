\section{Getting Started\label{sec:started}}
\markboth{\textsc{MLD2P4 User's and Reference Guide}}
         {\textsc{\ref{sec:started} Getting Started}}

We describe the basics for building and applying MLD2P4 one-level and multi-level
Schwarz preconditioners with the Krylov solvers included in PSBLAS \cite{PSBLASGUIDE}.
The following steps are required:
\begin{enumerate} 
\item \emph{Declare the preconditioner data structure}. It is a derived data type,
  \verb|mld_|\-\emph{x}\verb|prec_type|, where \emph{x} may be \verb|s|, \verb|d|, \verb|c|
	or \verb|z|, according to the basic data type of the sparse matrix
	(\verb|s| = real single precision; \verb|d| = real double precision;
	\verb|c| = complex single precision; \verb|z| = complex double precision).
	This data structure is accessed by the user only through the MLD2P4 routines,
	following an object-oriented approach.
\item \emph{Allocate and initialize the preconditioner data structure, according to
	a preconditioner type chosen by the user}. This is performed by the routine
	\verb|mld_precinit|, which also sets defaults for each preconditioner
	type selected by the user. The defaults associated to each preconditioner
	type are given in Table~\ref{tab:precinit}, where the strings used by
	\verb|mld_precinit| to identify the preconditioner types are also given.
	Note that these strings are valid also if uppercase letters are substituted by
	corresponding lowercase ones.
\item \emph{Modify the selected preconditioner type, by properly setting
  preconditioner parameters.} This is performed by the routine \verb|mld_precset|.
  This routine must be called only if the user wants to modify the default values
  of the parameters associated to the selected preconditioner type, to obtain a variant
  of the preconditioner. Examples of use of \verb|mld_precset| are given in
  Section~\ref{sec:examples}; a complete list of all the
  preconditioner parameters and their allowed and default values is provided in 
  Section~\ref{sec:userinterface}, Tables~\ref{tab:p_type}-\ref{tab:p_coarse}. 
\item \emph{Build the preconditioner for a given matrix.} This is performed by
  the routine \verb|mld_precbld|.
\item \emph{Apply the preconditioner at each iteration of a Krylov solver.}
  This is performed by the routine \verb|mld_precaply|. When using the PSBLAS Krylov solvers,
  this step is completely transparent to the user, since \verb|mld_precaply| is called
  by the PSBLAS routine implementing the Krylov solver (\verb|psb_krylov|).
\item \emph{Free the preconditioner data structure}. This is performed by
  the routine \verb|mld_precfree|. This step is complementary to step 1 and should
  be performed when the preconditioner is no more used.
\end{enumerate}
A detailed description of the above routines is given in Section~\ref{sec:userinterface}.
Examples showing the basic use of MLD2P4 are reported in Section~\ref{sec:examples}.

Note that the Fortran 95 module \verb|mld_prec_mod|, containing the definition of the 
preconditioner data type and the interfaces to the routines of MLD2P4,
must be used in any program calling such routines.
The modules \verb|psb_base_mod|, for the sparse matrix and communication descriptor
data types, and \verb|psb_krylov_mod|, for interfacing with the
Krylov solvers, must be also used (see Section~\ref{sec:examples}).

\ \\
\textbf{Remark 1.} The coarsest-level solver used by the default two-level
preconditioner has been chosen by taking into account that, on parallel
machines, it often leads to the smallest execution time when applied to
linear systems coming from finite-difference discretizations of basic
elliptic PDE problems, considered as standard tests for multi-level Schwarz
preconditioners \cite{aaecc_07,apnum_07}. However, this solver does
not necessarily correspond to the smallest number of iterations of the
preconditioned Krylov method, which is usually obtained by applying
a direct solver to the coarsest-level system, e.g.\ based on the LU
factorization (see Section~\ref{sec:userinterface}
for the coarsest-level solvers available in MLD2P4). 

\ \\
\textbf{Remark 2.} The include path for MLD2P4 must override
those for PSBLAS, e.g.\ the latter must come first in the sequence
passed to the compiler, as the MLD2P4 version of the Krylov solver
interfaces must override that of PSBLAS. This will change in the future
when the support for the \verb|class| statement becomes widespread in Fortran
compilers. 


\begin{table}[th]
\begin{center}
%{\small
\begin{tabular}{|l|l|p{6.4cm}|}
\hline
\textsc{type}       & \textsc{string} & \textsc{default preconditioner} \\ \hline
No preconditioner &\verb|'NOPREC'|& Considered only to use the PSBLAS
                                    Krylov solvers with no preconditioner. \\ \hline
Diagonal          & \verb|'DIAG'| & --- \\ \hline
Block Jacobi      & \verb|'BJAC'| & Block Jacobi with ILU(0) on the local blocks.\\ \hline
Additive Schwarz  & \verb|'AS'|   & Restricted Additive Schwarz (RAS),
                                    with overlap 1 and ILU(0) on the local blocks. \\ \hline
Multilevel        &\verb|'ML'|    & Multi-level hybrid preconditioner (additive on the
                                    same level and multiplicative through the levels),
                                    with post-smoothing only.
                                    Number of levels: 2.
	                                  Post-smoother: RAS with overlap 1 and ILU(0)
                                    on the local blocks.
                                    Aggregation: smoothed aggregation with
                                    threshold $\theta = 0$.
                                    Coarsest matrix: distributed among the processors.
                                    Coarsest-level solver: 
                                    4 sweeps of the block-Jacobi solver, 
                                    with LU factorization of the blocks
                                    (UMFPACK for the double precision versions and
                                    SuperLU for the single precision ones)         \\
\hline
\end{tabular}
%}
\end{center}

\caption{Preconditioner types, corresponding strings and default choices.
\label{tab:precinit}}
\end{table}

\subsection{Examples\label{sec:examples}}

The code reported in Figure~\ref{fig:ex_default} shows how to set and apply the default
multi-level preconditioner available in the real double precision version
of MLD2P4 (see Table~\ref{tab:precinit}). This preconditioner is chosen
by simply specifying \verb|'ML'| as second argument of \verb|mld_precinit|
(a call to \verb|mld_precset| is not needed) and is applied with the BiCGSTAB
solver provided by PSBLAS. As previously observed, the modules \verb|psb_base_mod|,
\verb|mld_prec_mod| and \verb|psb_krylov_mod| must be used by the example program.
 
The part of the code concerning the
reading and assembling of the sparse matrix and the right-hand side vector, performed
through the PSBLAS routines for sparse matrix and vector management, is not reported
here for brevity; the statements concerning the deallocation of the PSBLAS
data structure are neglected too.
The complete code can be found in the example program file \verb|mld_dexample_ml.f90|,
in the directory \verb|examples/fileread| of the MLD2P4 tree (see
Section~\ref{sec:building}).
For details on the use of the PSBLAS routines, see the PSBLAS User's
Guide \cite{PSBLASGUIDE}.

The setup and application of the default multi-level
preconditioners for the real single precision and the complex, single and double
precision, versions are obtained with straightforward modifications of the previous
example (see Section~\ref{sec:userinterface} for details). If these versions are installed,
the corresponding Fortran 95 codes are available in \verb|examples/fileread/|.

\begin{figure}[tbp]
\begin{center}
\begin{minipage}{.90\textwidth} 
{\small
\begin{verbatim}
  use psb_base_mod
  use mld_prec_mod
  use psb_krylov_mod
... ...
!
! sparse matrix
  type(psb_dspmat_type) :: A
! sparse matrix descriptor
  type(psb_desc_type)   :: desc_A
! preconditioner
  type(mld_dprec_type)  :: P
! right-hand side and solution vectors
  real(kind(1.d0))      :: b(:), x(:)
... ...
!
! initialize the parallel environment
  call psb_init(ictxt)
  call psb_info(ictxt,iam,np)
... ...
!
! read and assemble the matrix A and the right-hand side b 
! using PSBLAS routines for sparse matrix / vector management 
... ...
!
! initialize the default multi-level preconditioner, i.e. hybrid
! Schwarz, using RAS (with overlap 1 and ILU(0) on the blocks) 
! as post-smoother and 4 block-Jacobi sweeps (with UMFPACK LU
! on the blocks) as distributed coarse-level solver
  call mld_precinit(P,'ML',info)
!
! build the preconditioner
  call mld_precbld(A,desc_A,P,info)
!
! set the solver parameters and the initial guess
  ... ...
!
! solve Ax=b with preconditioned BiCGSTAB
  call psb_krylov('BICGSTAB',A,P,b,x,tol,desc_A,info)
  ... ...
!
! deallocate the preconditioner
  call mld_precfree(P,info)
!
! deallocate other data structures
  ... ...
!
! exit the parallel environment
  call psb_exit(ictxt)
  stop
\end{verbatim}
}
\end{minipage}
\caption{Setup and application of the default multi-level Schwarz preconditioner.
\label{fig:ex_default}}
\end{center}
\end{figure}

Different versions of multi-level preconditioners can be obtained by changing
the default values of the preconditioner parameters. The code reported in
Figure~\ref{fig:ex_3lh} shows how to set a three-level hybrid Schwarz
preconditioner, which uses block Jacobi with ILU(0) on the
local blocks as post-smoother, has a coarsest matrix replicated on the processors,
and solves the coarsest-level system with the LU factorization from UMFPACK~\cite{UMFPACK}.
The number of levels is specified by using \verb|mld_precinit|; the other
preconditioner parameters are set by calling \verb|mld_precset|. Note that
the type of multilevel framework (i.e.\ multiplicative among the levels
with post-smoothing only) is not specified since it is the default 
set by \verb|mld_precinit|. 

Figure~\ref{fig:ex_3la} shows how to
set a three-level additive Schwarz preconditioner,
which uses RAS, with overlap 1 and ILU(0) on the blocks, 
as pre- and post-smoother, and applies five block-Jacobi sweeps, with
the UMFPACK LU factorization on the blocks, as distributed coarsest-level
solver. Again, \verb|mld_precset| is used only to set
non-default values of the parameters (see Tables~\ref{tab:p_type}-\ref{tab:p_coarse}).
In both cases, the construction and the application of the preconditioner
are carried out as for the default multi-level preconditioner.
The code fragments shown in in Figures~\ref{fig:ex_3lh}-\ref{fig:ex_3la} are
included in the example program file \verb|mld_dexample_ml.f90| too.

Finally, Figure~\ref{fig:ex_1l} shows the setup of a one-level
additive Schwarz preconditioner, i.e.\ RAS with overlap 2. The corresponding
example program is available in \verb|mld_dexample_1lev.f90|.

For all the previous preconditioners, example programs where the sparse matrix and
the right-hand side are generated by discretizing a PDE with Dirichlet
boundary conditions are also available in the directory \verb|examples/pdegen|.


\begin{figure}[tbh]
\begin{center}
\begin{minipage}{.90\textwidth} 
{\small
\begin{verbatim}
... ...
! set a three-level hybrid Schwarz preconditioner, which uses 
! block Jacobi (with ILU(0) on the blocks) as post-smoother,
! a coarsest matrix replicated on the processors, and the 
! LU factorization from UMFPACK as coarse-level solver
  call mld_precinit(P,'ML',info,nlev=3)
  call_mld_precset(P,mld_smoother_type_,'BJAC',info)
  call mld_precset(P,mld_coarse_mat_,'REPL',info)
  call mld_precset(P,mld_coarse_solve_,'UMF',info)
... ...
\end{verbatim}
}
\end{minipage}

\caption{Setup of a hybrid three-level Schwarz preconditioner.\label{fig:ex_3lh}}
\end{center}
\end{figure}

\begin{figure}[tbh]
\begin{center}
\begin{minipage}{.90\textwidth} 
{\small
\begin{verbatim}
... ...
! set a three-level additive Schwarz preconditioner, which uses 
! RAS (with overlap 1 and ILU(0) on the blocks) as pre- and 
! post-smoother, and 5 block-Jacobi sweeps (with UMFPACK LU
! on the blocks) as distributed coarsest-level solver
  call mld_precinit(P,'ML',info,nlev=3)
  call mld_precset(P,mld_ml_type_,'ADD',info)
  call_mld_precset(P,mld_smoother_pos_,'TWOSIDE',info)
  call mld_precset(P,mld_coarse_sweeps_,5,info)
... ...
\end{verbatim}
}
\end{minipage}

\caption{Setup of an additive three-level Schwarz preconditioner.\label{fig:ex_3la}}
\end{center}
\end{figure}

\begin{figure}[tbh]
\begin{center}
\begin{minipage}{.90\textwidth} 
{\small
\begin{verbatim}
... ...
! set RAS with overlap 2 and ILU(0) on the local blocks
  call mld_precinit(P,'AS',info)
  call mld_precset(P,mld_sub_ovr_,2,info)
... ...
\end{verbatim}
}
\end{minipage}
\caption{Setup of a one-level Schwarz preconditioner.\label{fig:ex_1l}}
\end{center}
\end{figure}

\ \\
\textbf{Remark 3.} Any PSBLAS-based program using the basic preconditioners
implemented in PSBLAS 2.0, i.e.\ the diagonal and block-Jacobi ones,
can use the diagonal and block-Jacobi preconditioners
implemented in MLD2P4 without any change in the code.
The PSBLAS-based program must be only recompiled
and linked to the MLD2P4 library.

%%% Local Variables: 
%%% mode: latex
%%% TeX-master: "userguide"
%%% End: 
