\section{Configuring and Building MLD2P4\label{sec:building}}
\markboth{\textsc{MLD2P4 User's and Reference Guide}}
         {\textsc{\ref{sec:building} Configuring and Building MLD2P4}}
To build MLD2P4 it is necessary to set up a Makefile with appropriate
values for your system; this is done by means of the \verb|configure|
script. The distribution also includes the autoconf and automake
sources employed to generate the script, but usually this is not needed
to build the software. 

MLD2P4 is implemented almost entirely in Fortran~2003, with some
interfaces to external libraries in C; the Fortran compiler
must support the Fortran~2003 standard plus the extension \verb|MOLD=|
feature, which
enhances the usability of \verb|ALLOCATE|. 
Many compiles do this; in particular, this is
supported by the GNU Fortran compiler, for which we 
recommend to use at least version 4.7.2. 
The software defines data types and interfaces for
real and complex data, in both single and double precision. 

\subsection{Prerequisites}

The following base libraries are needed: 
\begin{description}
\item[BLAS] \cite{blas3,blas2,blas1} Many vendors provide optimized versions
  of the Basic Linear Algebra Subprograms; if no vendor version is
  available for a given platform, the ATLAS software
  (\verb!http://math-atlas.sourceforge.net/!)
  may be employed.  The reference BLAS from Netlib
  (\verb|http://www.netlib.org/blas|) are meant to define the standard
  behaviour of the BLAS interface, so they are not optimized for any
  particular plaftorm, and should only be used as a last
  resort. Note that BLAS computations form a relatively small part of
  the MLD2P4/PSBLAS computations; they are however critical when using
  preconditioners based on the UMFPACK or SuperLU third party
  libraries.  
\item[MPI] \cite{MPI2,MPI1} A version of MPI is available on most
  high-performance computing systems;
 \item[PSBLAS] \cite{PSBLASGUIDE,psblas_00} Parallel Sparse BLAS is
  available from \\ \verb|http://www.ce.uniroma2.it/psblas|; version
  3.3   (or later) is required. Indeed, all the prerequisites
  listed so far are also prerequisites of PSBLAS.
\end{description}
Please note that the four previous libraries must have Fortran
interfaces compatible with MLD2P4;
usually this means that they should all be built with the same
compiler as MLD2P4.

\subsection{Optional third party libraries}

We provide interfaces to the following third-party software libraries;
note that these are optional, but if you enable them some defaults
for multilevel preconditioners may change to reflect their presence. 

\begin{description}
\item[UMFPACK] \cite{UMFPACK}
  A sparse direct factorization package available from \\ 
  \verb|http://www.cise.ufl.edu/research/sparse/umfpack/|; 
  provides serial factorization and triangular system solution for double
  precision real and complex data. We have tested
  versions 5.4. 
\item[SuperLU] \cite{SUPERLU}
  A sparse direct factorization package available from \\
  \verb|http://crd.lbl.gov/~xiaoye/SuperLU/|; provides serial
  factorization and triangular system solution for single and double precision,
  real and complex data. We have tested version 4.3 and 5.0. 
 \item[SuperLU\_Dist] \cite{SUPERLUDIST}
   A sparse direct factorization package available
   from the same site as SuperLU; provides parallel factorization and
   triangular system solution for double precision real and complex data.
   We have tested version 3.3 and 4.2.
\end{description}
\subsection{Configuration options}

To build  MLD2P4 the first step is to use the \verb|configure| script
in the main directory to generate the necessary makefile(s). 

As a minimal example consider the following:
\begin{verbatim}
./configure --with-psblas=PSB-INSTALL-DIR
\end{verbatim}
which assumes that the various MPI compilers and support libraries are
available in the standard directories on the system, and specifies
only the PSBLAS install  directory (note that the latter directory must
be specified with an {\em absolute} path).
The full set of options may be looked at by issuing the command
\verb|./configure --help|, which produces:
\begin{verbatim}
`configure' configures MLD2P4 2.0 to adapt to many kinds of systems.

Usage: ./configure [OPTION]... [VAR=VALUE]...

To assign environment variables (e.g., CC, CFLAGS...), specify them as
VAR=VALUE.  See below for descriptions of some of the useful variables.

Defaults for the options are specified in brackets.

Configuration:
  -h, --help              display this help and exit
      --help=short        display options specific to this package
      --help=recursive    display the short help of all the included packages
  -V, --version           display version information and exit
  -q, --quiet, --silent   do not print `checking...' messages
      --cache-file=FILE   cache test results in FILE [disabled]
  -C, --config-cache      alias for `--cache-file=config.cache'
  -n, --no-create         do not create output files
      --srcdir=DIR        find the sources in DIR [configure dir or `..']

Installation directories:
  --prefix=PREFIX         install architecture-independent files in PREFIX
                          [/usr/local]
  --exec-prefix=EPREFIX   install architecture-dependent files in EPREFIX
                          [PREFIX]

By default, `make install' will install all the files in
`/usr/local/bin', `/usr/local/lib' etc.  You can specify
an installation prefix other than `/usr/local' using `--prefix',
for instance `--prefix=$HOME'.

For better control, use the options below.

Fine tuning of the installation directories:
  --bindir=DIR            user executables [EPREFIX/bin]
  --sbindir=DIR           system admin executables [EPREFIX/sbin]
  --libexecdir=DIR        program executables [EPREFIX/libexec]
  --sysconfdir=DIR        read-only single-machine data [PREFIX/etc]
  --sharedstatedir=DIR    modifiable architecture-independent data [PREFIX/com]
  --localstatedir=DIR     modifiable single-machine data [PREFIX/var]
  --libdir=DIR            object code libraries [EPREFIX/lib]
  --includedir=DIR        C header files [PREFIX/include]
  --oldincludedir=DIR     C header files for non-gcc [/usr/include]
  --datarootdir=DIR       read-only arch.-independent data root [PREFIX/share]
  --datadir=DIR           read-only architecture-independent data [DATAROOTDIR]
  --infodir=DIR           info documentation [DATAROOTDIR/info]
  --localedir=DIR         locale-dependent data [DATAROOTDIR/locale]
  --mandir=DIR            man documentation [DATAROOTDIR/man]
  --docdir=DIR            documentation root [DATAROOTDIR/doc/mld2p4]
  --htmldir=DIR           html documentation [DOCDIR]
  --dvidir=DIR            dvi documentation [DOCDIR]
  --pdfdir=DIR            pdf documentation [DOCDIR]
  --psdir=DIR             ps documentation [DOCDIR]

Optional Features:
  --disable-option-checking  ignore unrecognized --enable/--with options
  --disable-FEATURE       do not include FEATURE (same as --enable-FEATURE=no)
  --enable-FEATURE[=ARG]  include FEATURE [ARG=yes]
  --enable-serial         Specify whether to enable a fake mpi library to run
                          in serial mode.

Optional Packages:
  --with-PACKAGE[=ARG]    use PACKAGE [ARG=yes]
  --without-PACKAGE       do not use PACKAGE (same as --with-PACKAGE=no)
  --with-psblas=DIR       The install directory for PSBLAS, for example,
                          --with-psblas=/opt/packages/psblas-3.3
  --with-psblas-incdir=DIR
                          Specify the directory for PSBLAS includes.
  --with-psblas-libdir=DIR
                          Specify the directory for PSBLAS library.
  --with-extra-libs       List additional link flags here. For example,
                          --with-extra-libs=-lspecial_system_lib or
                          --with-extra-libs=-L/path/to/libs
  --with-umfpack=LIBNAME  Specify the library name for UMFPACK and its support
                          libraries. Default: "-lumfpack -lamd"
  --with-umfpackdir=DIR   Specify the directory for UMFPACK library and
                          includes.
  --with-umfpackincdir=DIR
                          Specify the directory for UMFPACK includes.
  --with-umfpacklibdir=DIR
                          Specify the directory for UMFPACK library.
  --with-superlu=LIBNAME  Specify the library name for SUPERLU library.
                          Default: "-lsuperlu"
  --with-superludir=DIR   Specify the directory for SUPERLU library and
                          includes.
  --with-superluincdir=DIR
                          Specify the directory for SUPERLU includes.
  --with-superlulibdir=DIR
                          Specify the directory for SUPERLU library.
  --with-superludist=LIBNAME
                          Specify the libname for SUPERLUDIST library.
                          Requires you also specify SuperLU. Default:
                          "-lsuperlu_dist"
  --with-superludistdir=DIR
                          Specify the directory for SUPERLUDIST library and
                          includes.
  --with-superludistincdir=DIR
                          Specify the directory for SUPERLUDIST includes.
  --with-superludistlibdir=DIR
                          Specify the directory for SUPERLUDIST library.

Some influential environment variables:
  FC          Fortran compiler command
  FCFLAGS     Fortran compiler flags
  LDFLAGS     linker flags, e.g. -L<lib dir> if you have libraries in a
              nonstandard directory <lib dir>
  LIBS        libraries to pass to the linker, e.g. -l<library>
  CC          C compiler command
  CFLAGS      C compiler flags
  CPPFLAGS    C/C++/Objective C preprocessor flags, e.g. -I<include dir> if
              you have headers in a nonstandard directory <include dir>
  CPP         C preprocessor
  MPICC       MPI C compiler command
  F77         Fortran 77 compiler command
  FFLAGS      Fortran 77 compiler flags
  MPIF77      MPI Fortran 77 compiler command
  MPIFC       MPI Fortran compiler command

Use these variables to override the choices made by `configure' or to help
it to find libraries and programs with nonstandard names/locations.

Report bugs to <bugreport@mld2p4.it>.
\end{verbatim}
For instance, if a user has built and installed PSBLAS 3.3 under the
\verb|/opt| directory and is
using the SuiteSparse package (which includes UMFPACK), then MLD2P4
might be configured with:
\begin{verbatim}
 ./configure --with-psblas=/opt/psblas-3.3/ \
 --with-umfpackincdir=/usr/include/suitesparse
\end{verbatim}
Once the configure script has completed execution, it will have
generated the file \verb|Make.inc| which will then be used by all
Makefiles in the directory tree; this file will be copied in the
install directory under the name \verb|Make.inc.MLD2P4|.

To build the library the user will now enter 
\begin{verbatim}
make
\end{verbatim}
followed (optionally) by 
\begin{verbatim}
make install
\end{verbatim}
\subsection{Bug reporting}
If you find any bugs in our codes, please let us know at
\begin{rawhtml}
<a href="mailto:bugreport@mld2p4.it">
\end{rawhtml}
\texttt{bugreport@mld2p4.it}
\begin{rawhtml}
</a>
\end{rawhtml}
; be aware that  
the amount of information needed to reproduce a problem in a parallel
program may vary quite a lot. 
\subsection{Example and test programs\label{sec:ex_and_test}}
The package contains the \verb|examples| and \verb|tests| directories;
both of them are further divided into \verb|fileread| and
\verb|pdegen| subdirectories. Their purpose is as follows:
\begin{description}
\item[\tt examples] contains a set of simple example programs with a
  predefined choice of preconditioners, selectable via integer
  values. These are intended to get an acquaintance with the
  multilevel preconditioners.
\item[\tt tests] contains a set of more sophisticated examples that
  will allow the user, via the input files in the \verb|runs|
  subdirectories, to experiment with the full range of preconditioners
  implemented in the library.
\end{description}
The \verb|fileread| directories contain sample programs that read
sparse matrices from files, according to the Matrix Market or the
Harwell-Boeing storage format; the \verb|pdegen| instead generate
matrices in full parallel mode from the discretization of a sample PDE. 
