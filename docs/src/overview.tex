\section{General Overview\label{sec:overview}}
\markboth{\textsc{MLD2P4 User's and Reference Guide}}
         {\textsc{\ref{sec:overview} General Overview}}
            
The \textsc{Multi-Level Domain Decomposition Parallel Preconditioners Package based on
PSBLAS (MLD2P4}) provides parallel Algebraic MultiGrid (AMG) and Domain 
Decomposition preconditioners (see, e.g., \cite{Briggs2000,Stuben_01,dd2_96}),
to be used in the iterative solution of  linear systems,
\begin{equation} 
Ax=b, 
\label{system1}
\end{equation} 
where $A$ is a square, real or complex, sparse matrix. 
%
%\textbf{NOTA: Caso non simmetrico, aggregazione con $(A+A^T)$ fatta! 
%Dovremmo implementare uno smoothed prolongator
%adeguato e fare qualcosa di consistente anche con 1-lev Schwarz.}
%
The name of the package comes from its original implementation, containing
multi-level additive and hybrid Schwarz preconditioners, as well as one-level additive
Schwarz preconditioners. The current version extends the original plan by including
multi-level cycles and smoothers widely used in multigrid methods.

The multi-level preconditioners implemented in MLD2P4 are obtained by combining
AMG cycles with smoothers and coarsest-level solvers. The V-, W-, and
K-cycles~\cite{Briggs2000,Notay2008} are available, which allow to define
almost all the preconditioners in the package, including the multi-level hybrid
Schwarz ones; a specific cycle is implemented to obtain multi-level additive
Schwarz preconditioners. The Jacobi, hybrid
%\footnote{see Note 2 in Table~\ref{tab:p_coarse}, p.~28.}
forward/backward Gauss-Seidel, block-Jacobi, and additive Schwarz methods
are available as smoothers. An algebraic approach is used to generate a hierarchy of
coarse-level matrices and operators, without explicitly using any information on the
geometry of the original problem, e.g., the discretization of a PDE. To this end,
the smoothed aggregation technique~\cite{BREZINA_VANEK,VANEK_MANDEL_BREZINA}
is applied. Either exact or approximate solvers can be used on the coarsest-level
system. Specifically, different sparse LU factorizations from external
packages, and native incomplete LU factorizations and Jacobi, hybrid Gauss-Seidel,
and block-Jacobi solvers are available. All smoothers can be also exploited as one-level 
preconditioners.

MLD2P4 is written in Fortran~2003, following an
object-oriented design through the exploitation of features
such as abstract data type creation, type extension, functional overloading, and
dynamic memory management. % \textbf{Va bene cos\'{i} o \`e meglio
% fare riferimento alle classi?}
The parallel implementation is based on a Single Program Multiple Data
(SPMD) paradigm.  Single and
double precision implementations of MLD2P4 are available for both the
real and the complex case, which can be used through a single
interface. 

MLD2P4 has been designed to implement scalable and easy-to-use
multilevel preconditioners in the context of the PSBLAS (Parallel Sparse BLAS)
computational framework~\cite{psblas_00,PSBLAS3}. PSBLAS provides basic linear algebra
operators and data management facilities for distributed sparse matrices,
as well as parallel Krylov solvers which can be used with the MLD2P4 preconditioners.
The choice of PSBLAS has been mainly motivated by the need of having
a portable and efficient software infrastructure implementing ``de facto'' standard
parallel sparse linear algebra kernels, to pursue goals such as performance,
portability, modularity ed extensibility in the development of the preconditioner
package. On the other hand, the implementation of MLD2P4 has led to some
revisions and extentions of the original PSBLAS kernels.
The inter-process comunication required by MLD2P4 is encapsulated
in the PSBLAS routines;
% , except few cases where MPI~\cite{MPI1} is explicitly called. 
therefore, MLD2P4 can be run on any parallel machine where PSBLAS 
implementations are available.

MLD2P4 has a layered and modular software architecture where three main layers can be
identified.  The lower layer consists of the PSBLAS kernels, the middle one implements
the construction and application phases of the preconditioners, and the upper one
provides a uniform interface to all the preconditioners. 
This architecture allows for different levels of use of the package:
few black-box routines at the upper layer allow all users to easily
build and apply any preconditioner available in MLD2P4;
facilities are also available allowing  expert users to extend the set of smoothers
and solvers for building new versions of the preconditioners (see
Section~\ref{sec:adding}). 

We note that the user interface of MLD2P4 2.1 has been extended with respect to the
previous versions in order to separate the construction of the multi-level hierarchy from
the construction of the smoothers and solvers, and to allow for more flexibility
at each level. The software architecture described in~\cite{MLD2P4_TOMS} has significantly
evolved too, in order to fully exploit the Fortran~2003 features implemented in PSBLAS 3.
However, compatibility with previous versions has been preserved.

This guide is organized as follows. General information on the distribution of the source
code is reported in Section~\ref{sec:distribution}, while details on the configuration
and installation of the package are given in Section~\ref{sec:building}. A short description
of the preconditioners implemented in MLD2P4 is provided in Section~\ref{sec:background},
to help the users in choosing among them. The basics for building and applying the
preconditioners with the Krylov solvers implemented in PSBLAS are reported 
in~Section~\ref{sec:started}, where the Fortran codes of a few sample programs
are also shown. A reference guide for the user interface routines is provided
in Section~\ref{sec:userinterface}. Information on the extension of the package
through the addition of new smoothers and solvers is reported in Section~\ref{sec:adding}. 
The error handling mechanism used by the package
is briefly described in Section~\ref{sec:errors}. The copyright terms concerning the
distribution and modification of MLD2P4 are reported in Appendix~\ref{sec:license}.

%%% Local Variables: 
%%% mode: latex
%%% TeX-master: "userguide"
%%% End: 
