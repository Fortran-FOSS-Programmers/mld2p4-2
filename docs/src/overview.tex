\section{General Overview\label{sec:overview}}
\markboth{\textsc{MLD2P4 User's and Reference Guide}}
         {\textsc{\ref{sec:overview} General Overview}}
            
The \textsc{Multi-Level Domain Decomposition Parallel Preconditioners Package based on
PSBLAS (MLD2P4}) provides parallel Algebraic MultiGrid (AMG) and domain decomposition 
preconditioners, designed to provide scalable and easy-to-use preconditioners 
multi-level Schwarz preconditioners~\cite{Stuben_01,dd2_96},
to be used in the iterative solutions of sparse linear systems:
\begin{equation} 
Ax=b, 
\label{system1}
\end{equation} 
where $A$ is a square, real or complex, sparse matrix. 
%
%\textbf{NOTA: Caso non simmetrico, aggregazione con $(A+A^T)$ fatta! 
%Dovremmo implementare uno smoothed prolongator
%adeguato e fare qualcosa di consistente anche con 1-lev Schwarz.}
%
Multi-level preconditioners can be obtained by combining several AMG cycles (V, W, K) with 
different smoothers (Jacobi, hybrid forward/backward Gauss-Seidel, block-Jacobi, additive Schwarz methods).
An algebraic approach is used to
generate a hierarchy of coarse-level matrices and operators, without
explicitly using any information on the geometry of the original problem, e.g.,
the discretization of a PDE. The smoothed aggregation technique is applied
as algebraic coarsening strategy~\cite{BREZINA_VANEK,VANEK_MANDEL_BREZINA}.
Either exact or approximate solvers are available to solve the coarsest-level system. Specifically,
different versions of sparse LU factorizations from external packages, and native incomplete
LU factorizations and iterative block-Jacobi solvers can be used.
All smoothers can be also exploited as one-level preconditioners.

MLD2P4 is written in Fortran~2003, following an
object-oriented design through the exploitation of features
such as abstract data type creation, functional overloading, and
dynamic memory management. 
The parallel implementation is based on a Single Program Multiple Data
(SPMD) paradigm.  Single and
double precision implementations of MLD2P4 are available for both the
real and the complex case, which can be used through a single
interface. 

MLD2P4 has been designed to implement scalable and easy-to-use
multilevel preconditioners in the context of the PSBLAS
(Parallel Sparse BLAS) computational framework~\cite{psblas_00,PSBLAS3}. 
PSBLAS provides basic linear algebra
operators and data management facilities for distributed sparse matrices,
as well as parallel Krylov solvers which can be coupled with the MLD2P4 preconditioners.
The choice of PSBLAS has been mainly motivated by the need of having
a portable and efficient software infrastructure implementing ``de facto'' standard
parallel sparse linear algebra kernels, to pursue goals such as performance,
portability, modularity ed extensibility in the development of the preconditioner
package. On the other hand, the implementation of MLD2P4 has led to some
revisions and extentions of the original PSBLAS kernels.
The inter-process comunication required
by MLD2P4 is encapsulated into the PSBLAS routines, except few cases where
MPI~\cite{MPI1} is explicitly called \textbf{\'E ancora cosi???}. Therefore, MLD2P4 can be run on any parallel
machine where PSBLAS and MPI implementations are available.

MLD2P4 has a layered and modular software architecture where three main layers can be identified. 
The lower layer consists of the PSBLAS kernels, the middle one implements
the construction and application phases of the preconditioners, and the upper one
provides a uniform interface to all the preconditioners. 
This architecture allows for different levels of use of the package:
few black-box routines at the upper layer allow non-expert users to easily
build any preconditioner available in MLD2P4 and to apply it within a PSBLAS Krylov solver;
{\bf facilities are also available that allow more expert users to extend the set of smoothers
and solvers for building new versions of preconditioners.}

We note that the user interface of MLD2P4 2.1 ({\bf Perche 2.1 e non 2.0???...Ricordarsi di cambiare il configure})
has been extended with respect to the previous versions 
in order to separate the construction
of the multi-level hierarchy from the construction of the smoothers and solvers, and to allow for more flexibility
at each level.
The software architecture described in~\cite{MLD2P4_TOMS} has significantly evolved too, in order to fully exploit the 
Fortran~2003 features implemented in PSBLAS 3.
However, compatibility with previous versions has been preserved.

This guide is organized as follows. General information on the distribution of the source code
is reported in Section~\ref{sec:distribution}, while details on the configuration
and installation of the package are given in Section~\ref{sec:building}. A short description of
the preconditioners implemented in MLD2P4 is provided
in Section~\ref{sec:background}, to help the users in choosing among them.
The basics for building and applying the preconditioners
with the Krylov solvers implemented in PSBLAS are reported in Section~\ref{sec:started}, where the
Fortran codes of a few sample programs are also shown. A reference guide for
the upper-layer routines of MLD2P4, that are the user interface, is provided
in Section~\ref{sec:userinterface}. The error handling mechanism used by the package is briefly described
in Section~\ref{sec:errors}. The copyright terms concerning the distribution and modification
of MLD2P4 are reported in Appendix~\ref{sec:license}.

%%% Local Variables: 
%%% mode: latex
%%% TeX-master: "userguide"
%%% End: 
