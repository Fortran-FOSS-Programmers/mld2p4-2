\section{User Interface\label{sec:userinterface}}
\markboth{\textsc{MLD2P4 User's and Reference Guide}}
         {\textsc{\ref{sec:userinterface} User Interface}}


The basic user interface of MLD2P4 consists of six routines. The four
routines \verb|mld_| \verb|precinit|, \verb|mld_precset|,
\verb|mld_precbld| and \verb|mld_precaply| encapsulate all the
functionalities for the setup and the application of any one-level and
multi-level preconditioner implemented in the package. 
The routine \verb|mld_precfree| deallocates the preconditioner data structure, while
\verb|mld_precdescr| prints a description of the preconditioner setup by the user.

For each routine, the same user interface is overloaded with
respect to the real/complex case and the single/double precision;
arguments with appropriate data types must be passed to the routine,
i.e.
\begin{itemize}
\item the sparse matrix data structure, containing the matrix to be
  preconditioned, must be of type \verb|psb_|\emph{x}\verb|spmat_type|
	with \emph{x} = \verb|s| for real single precision, \emph{x} = \verb|d|
	for real double precision, \emph{x} = \verb|c| for complex single precision,
	\emph{x} = \verb|z| for complex double precision;
\item the preconditioner data structure must be of type
  \verb|mld_|\emph{x}\verb|prec_type|, with \emph{x} =    
  \verb|s|, \verb|d|, \verb|c|, \verb|z|, according to the sparse
  matrix data structure;
\item the arrays containing the vectors $v$ and $w$ involved in
  the preconditioner application $w=M^{-1}v$ must be of type   
  \verb|psb_|\emph{x}\verb|vect_type| with \emph{x} =    
  \verb|s|, \verb|d|, \verb|c|, \verb|z|, in a manner completely
  analogous to the sparse matrix type;
\item real parameters defining the preconditioner must be declared
  according to the precision of the sparse matrix and preconditioner
  data structures (see Section~\ref{sec:precset}).
\end{itemize}
A description of each routine is given in the remainder of this section.

\clearpage

\subsection{Subroutine mld\_precinit\label{sec:precinit}}

\begin{center}
\verb|mld_precinit(p,ptype,info)| \\
\verb|mld_precinit(p,ptype,info,nlev)| \\
\end{center}

\noindent
This routine allocates and initializes the preconditioner data structure,
according to the preconditioner type chosen by the user.

{\vskip2\baselineskip\noindent\large\bfseries Arguments}

\begin{tabular}{p{1.2cm}p{12cm}}
\verb|p|      & \verb|type(mld_|\emph{x}\verb|prec_type), intent(inout)|.\\
              & The preconditioner data structure. Note that \emph{x}
                must be chosen according to the real/complex, single/double
                precision version of MLD2P4 under use.\\
\verb|ptype|  & \verb|character(len=*), intent(in)|.\\
              & The type of preconditioner. Its values are specified
              in Table~\ref{tab:precinit}.\\
              & Note that the strings are case insensitive.\\
\verb|info|   & \verb|integer, intent(out)|.\\
              & Error code. If no error, 0 is returned. See Section~\ref{sec:errors} for details.\\
\verb|nlev|   & \verb|integer, optional, intent(in)|.\\
              & The number of levels of the multilevel
                preconditioner. This optional argument is deprecated,
                new codes should set the number of levels with \verb|mld_precset|.\\
                % If \verb|nlev| is not present and \verb|ptype|=\verb|'ML'|, \verb|'ml'|, 
                % then \verb|nlev|=2 is assumed. Otherwise, \verb|nlev| is ignored.\\
\end{tabular}

\clearpage

\subsection{Subroutine mld\_precset\label{sec:precset}}

\begin{center}
\verb|call mld_precset(p,what,val,info)|
\end{center}

\noindent
This routine sets the parameters defining the preconditioner. More
precisely, the parameter identified by \verb|what| is assigned the value
contained in \verb|val|. 

The routine may also be invoked as a method
of the preconditioner object as in the following:
\begin{center}
\verb|call p%set(what,val,info [,ilev, ilmax, pos])|\\
\end{center}
In this case it is also possible to specify optional \verb|ilev| and
\verb|ilmax| arguments that restricts the effect of   
the call to the specified levels.

Finally, if the user has developed  a new type of smoother and/or
solver by extending one of the base MLD2P4 types, and has declared a
variable of the new  type in the main program, it is possible to pass
the new smoother/solver variable  to the setup routine as follows:
\begin{center}
\verb|call p%set(smoother,info [,ilev, ilmax,pos])|\\
\verb|call p%set(solver,info [,ilev, ilmax,pos])|
\end{center}
In this way, the variable will act as a \emph{mold} to which the
preconditioner will conform, even though the MLD2P4 library is not
modified, and thus has no direct knowledge about the new type. 

{\vskip2\baselineskip\noindent\large\bfseries Arguments}

\begin{tabular}{p{1.2cm}p{12cm}}
\verb|p|      & \verb|type(mld_|\emph{x}\verb|prec_type), intent(inout)|.\\
              & The preconditioner data structure. Note that \emph{x} must
                be chosen according to the real/complex, single/double precision
                 version of MLD2P4 under use.\\
\verb|what|   & \verb|integer, intent(in)| \emph{or} \verb|character(len=*)|. \\
              & The parameter to be set. It can be specified by 
                a predefined constant, or through its name; the string
                is case-insensitive. See also
                Tables~\ref{tab:p_type}-\ref{tab:p_coarse}.\\ 
\verb|val |   & \verb|integer| \emph{or} \verb|character(len=*)| \emph{or}
                \verb|real(psb_spk_)| \emph{or} \verb|real(psb_dpk_)|,
                \verb|intent(in)|.\\
              & The value of the parameter to be set. The list of allowed
                values and the corresponding data types is given in
                Tables~\ref{tab:p_type}-\ref{tab:p_coarse}.
                When the value is of type \verb|character(len=*)|,
                it is also treated as case insensitive.\\
\verb|smoother| & \verb|class(mld_x_base_smoother_type)| \\
              & The user-defined new smoother to be employed in the
                preconditioner.\\

\verb|solver| & \verb|class(mld_x_base_solver_type)| \\
              & The user-defined new solver to be employed in the
                preconditioner.\\

\verb|info|   & \verb|integer, intent(out)|.\\
              & Error code. If no error, 0 is returned. See Section~\ref{sec:errors}
                for details.\\
\verb|pos|   & \verb|charater(len=*), intent(in)|.\\
              & Whether the other arguments apply to the \verb|'PRE'|
                or to the \verb|'POST'| smoothers.\\

%
\verb|ilev|   & \verb|integer, optional, intent(in)|.\\
              & For the multilevel preconditioner, the level at which the
                preconditioner parameter has to be set.
                The levels are numbered in increasing
                order starting from the finest one, i.e.\ level 1 is the finest level.
                If \verb|ilev| is not present, the parameter identified by \verb|what|
                is set at all the appropriate levels (see
                Table~\ref{tab:params}).\\
\verb|ilmax|   & \verb|integer, optional, intent(in)|.\\
              & For the multilevel preconditioner, when both
                \verb|ilev| and \verb|ilmax| are present, the settings
                are applied at all levels \verb|ilev:ilmax|. When
                \verb|ilev| is present but \verb|ilmax| is not, then
                the default is \verb|ilmax=ilev|.
                The levels are numbered in increasing
                order starting from the finest one, i.e.\ level 1 is the finest level.
\end{tabular}

\ \\
A variety of (one-level and multi-level) preconditioners can be obtained
by a suitable setting of the preconditioner parameters. These parameters
can be logically divided into four groups, i.e.\ parameters defining
\begin{enumerate}
	\item the type of multi-level preconditioner;
	\item the one-level preconditioner used as smoother;
	\item the aggregation algorithm;
	\item the coarse-space correction at the coarsest level.
\end{enumerate}
A list of the parameters that can be set, along with their allowed and
default values, is given in Tables~\ref{tab:p_type}-\ref{tab:p_coarse}.
For a detailed description  of the meaning of the parameters, please
refer to Section~\ref{sec:background}. 

%
The smoother and solver objects are arranged in a hierarchical manner;
when specifying a smoother object, its parameters including the
contained solver are set to  default values, and when a solver
object is specified its defaults are also set, overriding in both
cases any previous settings even if explicitly specified. Therefore if
the user sets a new smoother, and wishes to use a solver
different from  the default one, the call to set the solver must come
\emph{after} the call to set the smoother. 
%

The combination of a Jacobi smoother with a Diagonal Scaling local
solver is equivalent to the strategy called Point Jacobi in the
literature; similarly, having a Jacobi smoother with a Gauss-Seidel
local solver is equivalent to a ``hybrid Gauss-Seidel'' solver. 


 Completely new smoother and/or  solver class derived from the
base objects in the library may be used without recompiling the
library itself. Once the new smoother/solver  class has been
developed, the user can declare a variable of that new type in the
application, and pass that variable to the \verb|p%set(solver,info)|
call; the new solver object is then dynamically included in the
preconditioner structure. 

The \verb|what,val| pairs described here are those of the predefined
smoother/solver objects; newly developed solvers may define new pairs
according to their needs. 


\bsideways
\begin{center}
\begin{tabular}{|p{5cm}|l|p{2cm}|l|p{7cm}|}
\hline
\verb|what|              & \textsc{data type}        &  \verb|val|      &  \textsc{default}  &
\textsc{comments} \\ \hline
%\multicolumn{5}{|c|}{\emph{type of the multi-level preconditioner}}\\ \hline
\verb|mld_ml_type_| \break \verb|ML_TYPE|     & \verb|character(len=*)|
                         & \texttt{'ADD'} \ \ \ \texttt{'MULT'}   
                         & \texttt{'MULT'}
                         & Basic multi-level framework: additive or multiplicative
                           among the levels (always additive inside a level).         \\ \hline 
\verb|mld_smoother_type_| \break \verb|SMOOTHER_TYPE|  & \verb|character(len=*)|
                         & \texttt{'JACOBI'} \ \ \ \texttt{'BJAC'} \ \ \ \texttt{'AS'}
                         & \texttt{'AS'}
                         & Basic predefined one-level preconditioner
                         (i.e.\ smoother): Jacobi, block Jacobi, AS. \\ \hline
\verb|mld_smoother_pos_| \break \verb|SMOOTHER_POS| & \verb|character(len=*)|
                         & \texttt{'PRE'} \ \ \ \texttt{'POST'} \ \ \ \texttt{'TWOSIDE'}
                         & \texttt{'TWOSIDE'}
                         & ``Position'' of the smoother: pre-smoother, post-smoother, 
                           pre- and post-smoother. \\
\hline
\end{tabular}
\end{center}
\caption{Parameters defining the type of multi-level preconditioner.
\label{tab:p_type}}                       
\esideways
                   
\bsideways
\begin{center}
\small
\begin{tabular}{|p{3.5cm}|l|p{3.2cm}|l|p{5cm}|}
\hline
\verb|what|              & \textsc{data type}        &  \verb|val|      &  \textsc{default}  &
\textsc{comments} \\ \hline
%\multicolumn{5}{|c|}{\emph{basic one-level preconditioner (smoother)}} \\ \hline
\verb|mld_sub_ovr_|  \break \verb|SUB_OVR|      & \verb|integer|
                         & any~int.~num.~$\ge 0$
                         & 1
                         & Number of overlap layers. \\ \hline
\verb|mld_sub_restr_|  \break \verb|SUB_RESTR|   & \verb|character(len=*)|
                         & \texttt{'HALO'} \hspace{2.5cm} \texttt{'NONE'}
                         & \texttt{'HALO'}
                         & Type of restriction operator:
                           \texttt{'HALO'} for taking into account the overlap, \texttt{'NONE'} 
                           for neglecting it. \\ \hline
\verb|mld_sub_prol_| \break \verb|SUB_PROL|   & \verb|character(len=*)|
                         & \texttt{'SUM'} \hspace{2.5cm} \texttt{'NONE'}
                         & \texttt{'NONE'}
                         & Type of prolongation operator:
                           \texttt{'SUM'} for adding the contributions from the overlap, \texttt{'NONE'}
                           for neglecting them.   \\ \hline
\verb|mld_sub_solve_| \break \verb|SUB_SOLVE|    & \verb|character(len=*)|
                         & \texttt{'DIAG'} \hspace{2.5cm}
                           \texttt{'GS'} \hspace{2.5cm} \texttt{'BWGS'} \hspace{2.5cm}
                           \texttt{'ILU'} \hspace{2.5cm} 
                           \texttt{'MILU'} \hspace{2.5cm} \texttt{'ILUT'} 
                           \hspace{2.5cm}
                           \texttt{'UMF'} \hspace{2.5cm}
                           \texttt{'SLU'}
                           \hspace{2.5cm} \texttt{'MUMPS'}
                         & \texttt{'ILU'}
                         & Predefined local solver: pointwise Jacobi
                           (diagonal scaling),
                           (forward) Gauss-Seidel, Backward
                           Gauss-Seidel, ILU($p$),  MILU($p$),
                           ILU($p,t$), LU from UMFPACK, LU  from 
                           SuperLU (plus triangular solve), LU from MUMPS. \\ \hline  
\verb|mld_sub_fillin_| \break \verb|SUB_FILLIN|  & \verb|integer|
                         & Any~int.~num.~$\ge 0$
                         & 0
                         & Fill-in level $p$ of the incomplete LU factorizations. \\ \hline
\verb|mld_sub_iluthrs_| \break \verb|SUB_ILUTHRS|  & \verb|real(|\emph{kind\_parameter}\verb|)|
                         & Any~real~num.~$\ge 0$
                         & 0
                         & Drop tolerance $t$ in the ILU($p,t$) factorization. \\ \hline
\verb|mld_sub_ren_| \break \verb|SUB_REN|   & \verb|character(len=*)|
                         & \texttt{'RENUM\_NONE'}  \texttt{'RENUM\_GLOBAL'} %, \texttt{'RENUM_GPS'}
                         & \texttt{'RENUM\_NONE'}
                         & Row and column reordering of the local submatrices: no reordering,
                           reordering according to the global numbering of the rows and columns of
                           the whole matrix. \\
\verb|mld_solver_sweeps_| \break \verb|SOLVER_SWEEPS|  & \verb|integer|
                         & Any~int.~num.~$\ge 1$
                         & 1
                         & Number of sweeps for iterative local solver
                           (currently only Gauss-Seidel). \\ \hline
% \verb|mld_solver_eps_| \break \verb|SOLVER_EPS|  & \verb|real|
%                          & Any~real~num.
%                          & 0
%                          & Stopping tolerance for iterative local solver
%                            (currently only Gauss-Seidel); if $\le0$, then
%                            perform prespecified number of iterations. \\ \hline
\hline
\end{tabular}
\end{center}
\caption{Parameters defining the one-level preconditioner used as smoother.
\label{tab:p_smoother}}  
\esideways
                   
\bsideways
\begin{center}
\begin{tabular}{|p{5cm}|l|p{2.4cm}|p{2.4cm}|p{6cm}|}
\hline
\verb|what|              & \textsc{data type}        &  \verb|val|      &  \textsc{default}  &
\textsc{comments} \\ \hline
%\multicolumn{5}{|c|}{\emph{aggregation algorithm}} \\ \hline
\verb|mld_coarse_aggr_size_|  \break \verb|COARSE_AGGR_SIZE| & \verb|integer|
                         & A positive number
                         &  The  cubic root of 
                           the matrix size at the fine level.
                         & Coarse size threshold. Aggregation will
                           proceed  until either the global number of
                           variables is below this threshold, or the
                           aggregation  does not reduce the size any
                           longer, or the maximum number of levels is
                           reached.  \\ \hline  
\verb|mld_min_aggr_ratio|  \break \verb|MIN_AGGR_RATIO| & \verb|real|
                         & A number greater than one
                         & 1.5
                         & Minimum aggregation ratio. Aggregation will
                           stop if the ratio between the matrix sizes
                           at two consecutive levels drops below this
                           threshold.\\ \hline   
\verb|mld_n_prec_levs_|  \break \verb|N_PREC_LEVS| & \verb|integer|
                         & A number greater than 1, or -1.
                         & -1
                         & Number of levels; if set to a positive
                           number greater than 1, it will force the
                           aggregation algorithm to use this many
                           levels (unless there is no reduction in the
                           coarse matrix size). \\ \hline  
\verb|mld_max_prec_levs_|  \break \verb|MAX_PREC_LEVS| & \verb|integer|
                         & A positive number
                         & 20
                         & Maximum number of levels: irrespective of
                           the other settings, do not use more than
                           this many levels. \\ \hline  
\verb|mld_aggr_alg_|  \break \verb|AGGR_ALG|  & \verb|character(len=*)|
                         & \texttt{'DEC'}, \texttt{'SYMDEC'}
                         & \texttt{'DEC'}
                         & Aggregation algorithm. Currently, only the
                         decoupled aggregation is available; the
                           \verb|SYMDEC| option  applies decoupled
                           aggregation to  the sparsity pattern
                           of $A+A^T$.\\ \hline 
\verb|mld_aggr_ord_|  \break \verb|AGGR_ORD|  & \verb|character(len=*)|
                         & \texttt{'NATURAL'}
                         & \texttt{'DEGREE'}
                         & Initial ordering of indices for aggregation
                            algorithm: natural ordering or sorted by
                            descending degree of the node in the
                            matrix graph. Since aggregation is
                            heuristics, results will be different. \\ \hline 
\hline
\end{tabular}
\end{center}
\caption{Parameters defining the aggregation algorithm.
\label{tab:p_aggregation}} 
\esideways

\bsideways
\begin{center}
\begin{tabular}{|p{5cm}|l|p{2.4cm}|p{2.4cm}|p{6cm}|}
\hline
\verb|what|              & \textsc{data type}        &  \verb|val|      &  \textsc{default}  &
\textsc{comments} \\ \hline
\verb|mld_aggr_kind_|  \break \verb|AGGR_KIND|  & \verb|character(len=*)|
                         & \texttt{'SMOOTHED'} \hspace{2.5cm} \texttt{'NONSMOOTHED'}
                         & \texttt{'SMOOTHED'}
                         & Type of aggregation: smoothed, nonsmoothed
                         (i.e.\ using the tentative prolongator). \\
  \hline
\verb|mld_aggr_thresh_| \break \verb|AGGR_THRESH| & \verb|real(|\emph{kind\_parameter}\verb|)|
                         & Any~real~num. $\in [0, 1]$
                         & 0.05
                         & Threshold $\theta$ in the aggregation algorithm. \\ \hline
\verb|mld_aggr_scale_| \break \verb|AGGR_SCALE| & \verb|real(|\emph{kind\_parameter}\verb|)|
                         & Any~real~num. $\in [0, 1]$
                         & 1.0
                         & Scale factor applied to the threshold going
                           from level $ilev$ to level $ilev+1$. \\ \hline
\verb|mld_aggr_omega_alg_|  \break \verb|AGGR_OMEGA_ALG|& \verb|character(len=*)|
                         & \texttt{'EIG\_EST'} \hspace{2.5cm} \texttt{'USER\_CHOICE'}
                         & \texttt{'EIG\_EST'}
                         & How the damping parameter $\omega$ in the
                           smoothed aggregation should be computed:
                           either via an estimate of the spectral radius of
                           $D^{-1}A$, or explicily
                           specified by the user. \\ \hline
\verb|mld_aggr_eig_|  \break \verb|AGGR_EIG|    & \verb|character(len=*)|
                         & \texttt{'A\_NORMI'}
                         & \texttt{'A\_NORMI'}
                         & How to estimate the spectral radius of $D^{-1}A$.
                           Currently only the infinity norm estimate
                           is available. \\ \hline
\verb|mld_aggr_omega_val_| \break \verb|AGGR_OMEGA_VAL|    & \verb|real(|\emph{kind\_parameter}\verb|)|
                         & Any~nonnegative~real~num.
                         & $4/(3\rho(D^{-1}A))$
                         & Damping parameter $\omega$ in the smoothed aggregation algorithm. 
                           It must be set by the user if
                           \verb|USER_CHOICE| was specified for 
                           \verb|mld_aggr_omega_alg_|,
                           otherwise it is computed by the library, using the
                           selected estimate of the spectral radius $\rho(D^{-1}A)$ of
                           $D^{-1}A$.\\
\hline
\end{tabular}
\end{center}
\caption{Parameters defining the aggregation algorithm.
\label{tab:p_aggregation_1}} 
\esideways
                     
\bsideways
\begin{center}
\begin{tabular}{|p{3.5cm}|l|p{3.2cm}|l|p{5cm}|}
\hline
\verb|what|              & \textsc{data type}        &  \verb|val|      &  \textsc{default}  &
\textsc{comments} \\ \hline
%\multicolumn{5}{|c|}{\emph{coarse-space correction at the coarsest level}}\\ \hline
\verb|mld_coarse_mat_|  \break \verb|COARSE_MAT|  & \verb|character(len=*)|
                         & \texttt{'DISTR'} \hspace{2.5cm} \texttt{'REPL'}
                         & \texttt{'DISTR'}
                         & Coarsest matrix: distributed among the processors or
                           replicated on each of them. \\ \hline
\verb|mld_coarse_solve_| \break \verb|COARSE_SOLVE| & \verb|character(len=*)|
                         & \texttt{'BJAC'} \hspace{2.5cm} 
                           \texttt{'UMF'} \hspace{2.5cm} \texttt{'MUMPS'} \hspace{2.5cm}
                           \texttt{'SLU'} \hspace{2.5cm} \texttt{'SLUDIST'}
                         & \texttt{'BJAC'}
                         & Solver used at the coarsest level: block Jacobi, sequential
                           LU from UMFPACK, sequential LU from SuperLU, 
                           distributed LU from SuperLU\_Dist or MUMPS.
                           \texttt{'SLUDIST'} and \texttt{'MUMPS'}
                           require the coarsest  matrix to be distributed, while
                           \texttt{'UMF'} and  \texttt{'SLU'} require
                           it to be replicated. \\ \hline 
\verb|mld_coarse_subsolve_| \break \verb|COARSE_SUBSOLVE| & \verb|character(len=*)|
                         & \texttt{'ILU'} \hspace{2.5cm} \texttt{'MILU'}
                           \hspace{2.5cm} \texttt{'ILUT'}
                           \hspace{2.5cm} \texttt{'UMF'}
                           \hspace{2.5cm} \texttt{'SLU'}
                           \hspace{2.5cm} \texttt{'MUMPS'}
                         & See note
                         & Solver for the diagonal blocks of the coarse matrix,
                           in case the block Jacobi solver
                           is chosen as coarsest-level solver: ILU($p$), MILU($p$),
                           ILU($p,t$), LU from UMFPACK,
                           LU from SuperLU, plus triangular solve. \\ \hline
\verb|mld_coarse_sweeps_| \break \verb|COARSE_SWEEPS| & \verb|integer|                         
                         & Any~int.~num.~$> 0$
                         & 4
                         & Number of Block-Jacobi sweeps when 'BJAC' is used as
                           coarsest-level solver. \\ \hline
\verb|mld_coarse_fillin_| \break \verb|COARSE_FILLIN| & \verb|integer|
                         & Any~int.~num.~$\ge 0$
                         & 0
                         & Fill-in level $p$ of the incomplete LU factorizations. \\ \hline
\verb|mld_coarse_iluthrs_|  \break \verb|COARSE_ILUTHRS| & \verb|real(|\emph{kind\_parameter}\verb|)|
                         & Any~real.~num.~$\ge 0$
                         & 0
                         & Drop tolerance $t$ in the ILU($p,t$) factorization. \\
\hline
\multicolumn{5}{|l|}{{\bfseries Note:} defaults for
  {\texttt mld\_coarse\_subsolve\_} are chosen as }\\ 
\multicolumn{5}{|l|}{single precision version: 'MUMPS' if installed, 'SLU' if installed, 'ILU' otherwise}\\
\multicolumn{5}{|l|}{double precision version: 'MUMPS' if installed, 'UMF' if installed,
  else 'SLU' if installed, 'ILU' otherwise}\\
\hline
\end{tabular}
\end{center}
\caption{Parameters defining the coarse-space correction at the coarsest
level.\label{tab:p_coarse}} 
\esideways

% \par\noindent{\large\bfseries Note}\par\noindent
% The defaults for parameter \verb|mld_coarse_subsolve_| in Table~\ref{tab:p_coarse} are determined
% as follows:
% \begin{description}
% \item[Single precision data:] 'SLU' if installed, 'ILU' otherwise;
% \item[Double precision data:] 'UMF' if installed, else 'SLU' if
%   installed, 'ILU' otherwise;
% \end{description}


\clearpage

\subsection{Subroutine mld\_precbld\label{sec:precbld}}
  
\begin{center}
\verb|mld_precbld(a,desc_a,p,info)|\\
\end{center}

\noindent
This routine builds the preconditioner according to the requirements made by
the user through the routines \verb|mld_precinit| and \verb|mld_precset|.

For multilevel preconditioner this routine is supported for backward
compatibility, but we recommend to use the routines of
Sec.~\ref{sec:hier_bld} and~\ref{sec:smoothers_bld}.
{\vskip2\baselineskip\noindent\large\bfseries Arguments}

\begin{tabular}{p{1.2cm}p{12cm}}
\verb|a|      & \verb|type(psb_|\emph{x}\verb|spmat_type), intent(in)|. \\
              & The sparse matrix structure containing the local part of the
                matrix to be preconditioned. Note that \emph{x} must be chosen according
                to the real/complex, 
single/double precision version of MLD2P4 under use.
                See the PSBLAS User's Guide for details \cite{PSBLASGUIDE}.\\
\verb|desc_a| & \verb|type(psb_desc_type), intent(in)|. \\
              & The communication descriptor of \verb|a|. See the PSBLAS User's Guide for
                details \cite{PSBLASGUIDE}.\\
\verb|p|      & \verb|type(mld_|\emph{x}\verb|prec_type), intent(inout)|.\\
              & The preconditioner data structure. Note that \emph{x} must be chosen according
                to the real/complex, single/double precision version of MLD2P4 under use.\\
\verb|info|   & \verb|integer, intent(out)|.\\
              & Error code. If no error, 0 is returned. See Section~\ref{sec:errors} for details.\\
\end{tabular}

\subsection{Subroutine mld\_hierarchy\_bld\label{sec:hier_bld}}
  
\begin{center}
\verb|mld_hierachy_bld(a,desc_a,p,info)|\\
\end{center}

\noindent
This routine builds the aggregation hierarchy  according to the requirements made by
the user through the routines \verb|mld_precinit| and \verb|mld_precset|.

{\vskip2\baselineskip\noindent\large\bfseries Arguments}

\begin{tabular}{p{1.2cm}p{12cm}}
\verb|a|      & \verb|type(psb_|\emph{x}\verb|spmat_type), intent(in)|. \\
              & The sparse matrix structure containing the local part of the
                matrix to be preconditioned. Note that \emph{x} must be chosen according
                to the real/complex, 
single/double precision version of MLD2P4 under use.
                See the PSBLAS User's Guide for details \cite{PSBLASGUIDE}.\\
\verb|desc_a| & \verb|type(psb_desc_type), intent(in)|. \\
              & The communication descriptor of \verb|a|. See the PSBLAS User's Guide for
                details \cite{PSBLASGUIDE}.\\
\verb|p|      & \verb|type(mld_|\emph{x}\verb|prec_type), intent(inout)|.\\
              & The preconditioner data structure. Note that \emph{x} must be chosen according
                to the real/complex, single/double precision version of MLD2P4 under use.\\
\verb|info|   & \verb|integer, intent(out)|.\\
              & Error code. If no error, 0 is returned. See Section~\ref{sec:errors} for details.\\
\end{tabular}


\subsection{Subroutine mld\_smoothers\_bld\label{sec:smoothers_bld}}
  
\begin{center}
\verb|mld_smoothers_bld(a,desc_a,p,info)|\\
\end{center}

\noindent
This routine builds the preconditioner according to the requirements made by
the user through the routines \verb|mld_precinit| and
\verb|mld_precset|, based on the aggregation hierahy produced by a
previous call to \verb|mld_hierarchy_bld| (see
Sec.~\ref{sec:hier_bld}). 

{\vskip2\baselineskip\noindent\large\bfseries Arguments}

\begin{tabular}{p{1.2cm}p{12cm}}
\verb|a|      & \verb|type(psb_|\emph{x}\verb|spmat_type), intent(in)|. \\
              & The sparse matrix structure containing the local part of the
                matrix to be preconditioned. Note that \emph{x} must be chosen according
                to the real/complex, 
single/double precision version of MLD2P4 under use.
                See the PSBLAS User's Guide for details \cite{PSBLASGUIDE}.\\
\verb|desc_a| & \verb|type(psb_desc_type), intent(in)|. \\
              & The communication descriptor of \verb|a|. See the PSBLAS User's Guide for
                details \cite{PSBLASGUIDE}.\\
\verb|p|      & \verb|type(mld_|\emph{x}\verb|prec_type), intent(inout)|.\\
              & The preconditioner data structure. Note that \emph{x} must be chosen according
                to the real/complex, single/double precision version of MLD2P4 under use.\\
\verb|info|   & \verb|integer, intent(out)|.\\
              & Error code. If no error, 0 is returned. See Section~\ref{sec:errors} for details.\\
\end{tabular}

\clearpage
\subsection{Subroutine mld\_precaply\label{sec:precaply}}

\begin{center}
\verb|mld_precaply(p,x,y,desc_a,info)|\\
\verb|mld_precaply(p,x,y,desc_a,info,trans,work)|\\
\end{center}

\noindent
This routine computes $y = op(M^{-1})\, x$, where $M$ is a previously built
preconditioner, stored into \verb|p|, and $op$
denotes the preconditioner itself or its transpose, according to
the value of \verb|trans|.
Note that, when MLD2P4 is used with a Krylov solver from PSBLAS,
\verb|mld_precaply| is called within the PSBLAS routine \verb|psb_krylov|
and hence it is completely transparent to the user.

{\vskip2\baselineskip\noindent\large\bfseries Arguments}

\begin{tabular}{p{1.2cm}p{12cm}}
\verb|p|      & \verb|type(mld_|\emph{x}\verb|prec_type), intent(inout)|.\\
              & The preconditioner data structure, containing the local part of $M$.
                Note that \emph{x} must be chosen according
                to the real/complex, single/double precision version of MLD2P4 under use.\\
\verb|x|      & \emph{type}\verb|(|\emph{kind\_parameter}\verb|), dimension(:), intent(in)|.\\
              & The local part of the vector $x$. Note that \emph{type} and   
                \emph{kind\_parameter} must be chosen according
                to the real/complex, single/double precision version of MLD2P4 under use.\\
\verb|y|      & \emph{type}\verb|(|\emph{kind\_parameter}\verb|), dimension(:), intent(out)|.\\
              & The local part of the vector $y$. Note that \emph{type} and
                \emph{kind\_parameter} must be chosen according
                to the real/complex, single/double precision version of MLD2P4 under use.\\
\verb|desc_a| & \verb|type(psb_desc_type), intent(in)|. \\
              & The communication descriptor associated to the matrix to be
                preconditioned.\\
\verb|info|   & \verb|integer, intent(out)|.\\
              & Error code. If no error, 0 is returned. See Section~\ref{sec:errors} for details.\\
\verb|trans|  & \verb|character(len=1), optional, intent(in).|\\
              & If \verb|trans| = \verb|'N','n'| then $op(M^{-1}) = M^{-1}$;
                if \verb|trans| = \verb|'T','t'| then $op(M^{-1}) = M^{-T}$
                (transpose of $M^{-1})$;  if \verb|trans| = \verb|'C','c'| then $op(M^{-1}) = M^{-C}$
                (conjugate transpose of $M^{-1})$.\\
\verb|work|  & \emph{type}\verb|(|\emph{kind\_parameter}\verb|), dimension(:), optional, target|.\\
             & Workspace. Its size should be at
               least \verb|4 * psb_cd_get_local_| \verb|cols(desc_a)| (see the PSBLAS User's Guide).
               Note that \emph{type} and \emph{kind\_parameter} must be chosen according
               to the real/complex, single/double precision version of MLD2P4 under use.\\
\end{tabular}

\clearpage

\subsection{Subroutine mld\_precfree\label{sec:precfree}}

\begin{center}
\verb|mld_precfree(p,info)|\\
\end{center}

\noindent
This routine deallocates the preconditioner data structure.

{\vskip2\baselineskip\noindent\large\bfseries Arguments}

\begin{tabular}{p{1.2cm}p{10.5cm}}
\verb|p|      & \verb|type(mld_|\emph{x}\verb|prec_type), intent(inout)|.\\
              & The preconditioner data structure. Note that \emph{x} must be chosen according
                to the real/complex, single/double precision version of MLD2P4 under use.\\
\verb|info|   & \verb|integer, intent(out)|.\\
              & Error code. If no error, 0 is returned. See Section~\ref{sec:errors} for details.\\
\end{tabular}

\clearpage

\subsection{Subroutine mld\_precdescr\label{sec:precdescr}}

\begin{center}
\verb|mld_precdescr(p,info)|\\
\verb|mld_precdescr(p,info,iout)|\\
\end{center}

\noindent
This routine prints a description of the preconditioner to the standard output or
to a file. It must be called after \verb|mld_precbld| has been called.

{\vskip2\baselineskip\noindent\large\bfseries Arguments}

\begin{tabular}{p{1.2cm}p{12cm}}
\verb|p|      & \verb|type(mld_|\emph{x}\verb|prec_type), intent(in)|.\\
              & The preconditioner data structure. Note that \emph{x} must be chosen according
                to the real/complex, single/double precision version of MLD2P4 under use.\\
\verb|info|   & \verb|integer, intent(out)|.\\
              & Error code. If no error, 0 is returned. See Section~\ref{sec:errors} for details.\\
\verb|iout|   & \verb|integer, intent(in), optional|.\\
              & The id of the file where the preconditioner description
                will be printed; the default is the standard output.\\
\end{tabular}

%%% Local Variables: 
%%% mode: latex
%%% TeX-master: "userguide"
%%% End: 
